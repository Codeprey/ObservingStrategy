% ====================================================================
%+
% SECTION:
%    SolarSystem_OrbitalAccuracy.tex
%
% CHAPTER:
%    solarsystem.tex
%
% ELEVATOR PITCH:
%    How secure is the orbit - is it going to hit us?
%    Libration amplitude distribution for TNOs?
%    Can we find it after X years for further study?
%    Can we identify the source region for NEOs within the main belt?
%-
% ====================================================================

\section{Orbital Accuracy}
\def\secname{\chpname:orbits}\label{sec:\secname}

\credit{rhiannonlynne},
\credit{davidtrilling}

A vast number of moving objects will appear in LSST images. Multiple
observations of a common object will be linked, and a preliminary orbit
derived. However, the orbital elements (semi-major axis, eccentricity,
etc.) will have some uncertainty. Short arcs --- that is, a small amount
of time between the first and last observation of a given object ---
produce orbits with large uncertainties on the orbital elements. As arc
length grows, the orbital uncertainties decrease.

A number of science cases require relatively small uncertainties on
orbital elements. Perhaps most importantly, small uncertainties can aid
in discriminating between Near Earth Objects that might and might not
impact the Earth. A more subtle example relates to the libration
amplitude distribution for TNOs, which can be compared to predictions
from Solar System formation models. Only with small uncertainties on
orbital elements can the libration amplitudes be determined to
sufficient precision to compare to the predictive models. Finally,
during and after the primary LSST survey additional measurements will be
desired for further characterization of many objects. Only if the
orbital elements are sufficiently well known can objects be studied later
with other facilities. For example, to carry out spectroscopy, the
position of the object must be known to approximately 1~arcsec (the
width of a typical slit). This places strong requirements on the
knowledge of the orbital elements.


% --------------------------------------------------------------------

\subsection{Target measurements and discoveries}
\label{sec:\secname:targets}

The relevant data here are positions as a function of time for a given
object (assuming that the linking of measurements to a given object is
satisfactory). Assuming that the accuracy and precision of each
measurement are approximately constant (likely, since all will be made
by the same observing system), the only significant factor that improves
the knowledge of the orbit is extending the observational arc. The
observing strategy employed by LSST must therefore have a cadence in
which objects are revisited with the largest possible arcs that still
allow linking of observations. In other words, if the observations of a
given object are too widely spaced, linking may not be possible, so,
even though the arc is long, the linking is poor and the object yield is
low. If the observations are made too densely in time, linking is likely
to be good, but the arc may not be very long. A middle ground is
desired.

% --------------------------------------------------------------------

\subsection{Metrics}
\label{sec:\secname:metrics}

The best metric here would be to take the actual series of observations
of each object, add appropriate astrometric noise to each observation
according to its SNR, cull observations which would not be `linkable' to
the rest (i.e.\ observations which occur on a single night far from other
nights in the arc, or even a series of observations which occur too many
years away from other observations of the same object), and then fit an
orbit to the remaining observations and determine the uncertainty in its
parameters. This is work for the future however; our first simple proxy
uses the {\tt ObsArcMetric} to just look at the time between the first
and last observation of an object. For many objects, this will be fairly
close to the actual arc length of the linkable observations, as most
objects receive many observations clumped together when they are
observable, so this simple proxy makes a reasonable starting point.


% --------------------------------------------------------------------

\subsection{OpSim Analysis}
\label{sec:\secname:analysis}

\begin{figure}
\includegraphics[width=6in]{figs/solarsystem/minion_1016_ObsArc_neo_tno_mba_MOOB_ComboMetricVsH}
\caption{Mean observational arc length, in years, for NEO, MBA and TNO
  populations as a function of $H$ magnitude.
\label{obsarc}}
\end{figure}

In \opsimdbref{db:baseCadence}, the mean observational arc length for
NEOs and MBAs is about 8~years for bodies larger than 1~km, and about
6~years for 300~m bodies. With these
orbital arcs, the orbits will be quite well known,
% xxx quantify -- how well! xxx,
meaning that the majority
of LSST-observed objects will have orbits that are sufficiently
well known that the above science cases can be carried out.
In some special cases --- for example, the case where
an NEO's orbit still presents a significant probability
of terrestrial impact --- additional non-LSST follow-up
may be needed, but this will be a small minority of cases.

% xxx what we probably want is fraction of objects
% with arcs longer than 3 years (say) as a fxn of
% H mag, for NEOs/MBAs/TNOs. xxx

% --------------------------------------------------------------------

\subsection{Discussion}
\label{sec:\secname:discussion}

The simple proxy metric above should be improved to account for
potential difficulties in linking observations, and to include actual
orbital fitting to determine orbital uncertainties. The timing of
observations effects the final orbital accuracy significantly,
particularly for TNOs, and having a good distribution on the times of
observations can improve orbital accuracy more quickly than would
naively be expected from a simple observational arclength scaling.

A figure of merit, including requirements on the orbital accuracy for
various classes of objects, should also be developed.

As an intermediate step, we have carried out two anecdotal studies
of orbital accuracy. In the first,
we took an arbitrary (real) NEO --- object 2016~DL ---
with a six year arc. This is representative
and typical of NEOs that will be observed
by LSST.
The maximum positional uncertainty for this object
over the next ten years
is 25~arcsec (3$\sigma$). This is small enough
that essentially any kind of follow-up observation would be
possible (presuming that pre-imaging is possible
for spectroscopy, for example, to locate the moving
object).
The uncertainties on the orbital parameters
of this object are on the order of
1 part in 10$^4$ or even smaller. This level of
precision should allow all the investigations described
above.

In the second experiment, we took an arbitrary
(real) TNO --- object 2015~SO20 ---
that has a five year arc.
For this object, the maximum positional uncertainty
over the next ten years is $\sim$20'', and its orbital
elements are known to
around 1 part in 10$^5$. Again, this level
of precision should enable all of the science
investigations described above.


% ====================================================================
%
% \subsection{Conclusions}
%
% Here we answer the ten questions posed in
% \autoref{sec:intro:evaluation:caseConclusions}:
%
% \begin{description}
%
% \item[Q1:] {\it Does the science case place any constraints on the
% tradeoff between the sky coverage and coadded depth? For example, should
% the sky coverage be maximized (to $\sim$30,000 deg$^2$, as e.g., in
% Pan-STARRS) or the number of detected galaxies (the current baseline 
% of 18,000 deg$^2$)?}
%
% \item[A1:] ...
%
% \item[Q2:] {\it Does the science case place any constraints on the
% tradeoff between uniformity of sampling and frequency of  sampling? For
% example, a rolling cadence can provide enhanced sample rates over a part
% of the survey or the entire survey for a designated time at the cost of
% reduced sample rate the rest of the time (while maintaining the nominal
% total visit counts).}
%
% \item[A2:] ...
%
% \item[Q3:] {\it Does the science case place any constraints on the
% tradeoff between the single-visit depth and the number of visits
% (especially in the $u$-band where longer exposures would minimize the
% impact of the readout noise)?}
%
% \item[A3:] ...
%
% \item[Q4:] {\it Does the science case place any constraints on the
% Galactic plane coverage (spatial coverage, temporal sampling, visits per
% band)?}
%
% \item[A4:] ...
%
% \item[Q5:] {\it Does the science case place any constraints on the
% fraction of observing time allocated to each band?}
%
% \item[A5:] ...
%
% \item[Q6:] {\it Does the science case place any constraints on the
% cadence for deep drilling fields?}
%
% \item[A6:] ...
%
% \item[Q7:] {\it Assuming two visits per night, would the science case
% benefit if they are obtained in the same band or not?}
%
% \item[A7:] ...
%
% \item[Q8:] {\it Will the case science benefit from a special cadence
% prescription during commissioning or early in the survey, such as:
% acquiring a full 10-year count of visits for a small area (either in all
% the bands or in a  selected set); a greatly enhanced cadence for a small
% area?}
%
% \item[A8:] ...
%
% \item[Q9:] {\it Does the science case place any constraints on the
% sampling of observing conditions (e.g., seeing, dark sky, airmass),
% possibly as a function of band, etc.?}
%
% \item[A9:] ...
%
% \item[Q10:] {\it Does the case have science drivers that would require
% real-time exposure time optimization to obtain nearly constant
% single-visit limiting depth?}
%
% \item[A10:] ...
%
% \end{description}

% --------------------------------------------------------------------

\navigationbar
