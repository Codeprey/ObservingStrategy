% ====================================================================
%+
% SECTION:
%    MCs_ProperMotion.tex
%
% CHAPTER:
%    magclouds.tex
%
% ELEVATOR PITCH:
%-
% ====================================================================

% \section{The Proper Motion of the LMC and SMC}
\subsection{Exoplanets in the LMC and SMC}
\def\secname{\chpname:MC_exoplanets}\label{sec:\secname}

\credit{lundmb},
\credit{migueldvb}

While exoplanets are discussed in greater depth in \autoref{sec:planets}, it is
also worth noting here the unique circumstance of exoplanets in the
Magellanic Clouds. To date, all detected exoplanets have been found around
host stars within the Milky Way. Any constraints that could be applied to
planet occurrence rates in such a different stellar population as is found
in the Magellanic Clouds would provide a fresh insight into the limits
that are to be placed on planet formation rates.

The transit method of exoplanet detection is constrained by sufficient
period coverage in the observations taken, and in the dimming caused by
the star's transit being large enough with respect to the noise in
observations that the periodic signal of the transit can be recovered.
The relatively small chance of a planet being present and properly
aligned is offset by observing a large number of stars simultaneously.
Simulations have already shown that LSST has the capability to recover
the correct periods for large exoplanets around stars at the distance of
the LMC \citet{2015AJ....149...16L}.  We note that it would be unlikely
to be able to conduct follow-up observations of the discovered
candidates to confirm their planetary nature at a distance of $\sim$
50 kpc.  Further work is needed to characterize the ability to detect
these planets with sufficiently significant power to determine the
planet yield that could be expected from the LMC (Lund et al. in prep).


% --------------------------------------------------------------------

% \subsection{Metrics}
\subsubsection{Metrics}
\label{sec:\secname:metrics}

The case of transiting exoplanets in the Magellanic Clouds will benefit
from the same metrics that are used by transiting exoplanets within the
Milky Way, and are addressed in \autoref{sec:variables:variablemetrics}
and \autoref{sec:planets}. The key properties of the OpSim to be
measured will be those that relate to the number of observations that
will be made during planetary transits, and the overall phase coverage
of observations.  Unlike the general case of transiting planets in LSST,
transiting planets in the Magellanic Clouds specifically will likely
only have any meaning in deep-drilling fields, or some other comparable
cadence.

% % --------------------------------------------------------------------
%
% \subsection{Metrics}
% \label{sec:\secname:metrics}
%
% % --------------------------------------------------------------------
%
% \subsection{OpSim Analysis}
% \label{sec:\secname:analysis}
%
% % --------------------------------------------------------------------
%
% \subsection{Discussion}
% \label{sec:\secname:discussion}
%
% ====================================================================
%
% \subsection{Conclusions}
%
% Here we answer the ten questions posed in
% \autoref{sec:intro:evaluation:caseConclusions}:
%
% \begin{description}
%
% \item[Q1:] {\it Does the science case place any constraints on the
% tradeoff between the sky coverage and coadded depth? For example, should
% the sky coverage be maximized (to $\sim$30,000 deg$^2$, as e.g., in
% Pan-STARRS) or the number of detected galaxies (the current baseline 
% of 18,000 deg$^2$)?}
%
% \item[A1:] ...
%
% \item[Q2:] {\it Does the science case place any constraints on the
% tradeoff between uniformity of sampling and frequency of  sampling? For
% example, a rolling cadence can provide enhanced sample rates over a part
% of the survey or the entire survey for a designated time at the cost of
% reduced sample rate the rest of the time (while maintaining the nominal
% total visit counts).}
%
% \item[A2:] ...
%
% \item[Q3:] {\it Does the science case place any constraints on the
% tradeoff between the single-visit depth and the number of visits
% (especially in the $u$-band where longer exposures would minimize the
% impact of the readout noise)?}
%
% \item[A3:] ...
%
% \item[Q4:] {\it Does the science case place any constraints on the
% Galactic plane coverage (spatial coverage, temporal sampling, visits per
% band)?}
%
% \item[A4:] ...
%
% \item[Q5:] {\it Does the science case place any constraints on the
% fraction of observing time allocated to each band?}
%
% \item[A5:] ...
%
% \item[Q6:] {\it Does the science case place any constraints on the
% cadence for deep drilling fields?}
%
% \item[A6:] ...
%
% \item[Q7:] {\it Assuming two visits per night, would the science case
% benefit if they are obtained in the same band or not?}
%
% \item[A7:] ...
%
% \item[Q8:] {\it Will the case science benefit from a special cadence
% prescription during commissioning or early in the survey, such as:
% acquiring a full 10-year count of visits for a small area (either in all
% the bands or in a  selected set); a greatly enhanced cadence for a small
% area?}
%
% \item[A8:] ...
%
% \item[Q9:] {\it Does the science case place any constraints on the
% sampling of observing conditions (e.g., seeing, dark sky, airmass),
% possibly as a function of band, etc.?}
%
% \item[A9:] ...
%
% \item[Q10:] {\it Does the case have science drivers that would require
% real-time exposure time optimization to obtain nearly constant
% single-visit limiting depth?}
%
% \item[A10:] ...
%
% \end{description}
%
% ====================================================================
%
% \navigationbar
