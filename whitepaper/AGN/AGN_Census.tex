% ====================================================================
%+
% SECTION:
%    AGN_Census.tex
%
% CHAPTER:
%    agn.tex
%
% ELEVATOR PITCH:
%-
% ====================================================================

% \section{AGN Selection and Census}
\section{AGN Selection and Census}\label{sec:AGNCensus}
\def\secname{\chpname:census}\label{sec:\secname}

\credit{ohadshemmer},
\credit{nielbrandt},
\credit{GordonRichards},
\credit{AstroVPK},
\credit{ScottAnderson}

One basic figure of merit for AGN science is the total number of AGNs
discovered in the entire LSST survey, as a function of luminosity and
redshift. The main goal is therefore to adjust the Observing Strategy
in order to maximize this number.
%The primary goal for AGN science is to maximize the discovery of AGN
%with the LSST and construct the largest possible inventory of sources
%spanning the widest possible ranges in the redshift-luminosity parameter
%space. This, in turn, will provide
Doing so will provide tighter constraints within the context
of various cosmological science cases, such as quasar clustering,
$z>6$ quasars and reionization, and strong gravitational lensing.

% --------------------------------------------------------------------

% \subsection{Target measurements and discoveries}
\subsection{Target measurements and discoveries}
\label{sec:\secname:targets}

It is expected that $\approx 10^7 - 10^8$ AGNs will be selected in the
main LSST survey using a combination of criteria, split broadly into
four categories: colors, astrometry, variability, and multiwavelength
matching with other surveys \citep{2009arXiv0912.0201L, 2013AAS...22124710S}.
The LSST Observing strategy will mostly affect the first three of these
categories as described further below.

{\bf Colors:} The LSST observing strategy will determine the depth in each band,
as a function of position on the sky, and will thus affect the color selection
of AGNs. Additionally, it will affect the reliability of the actual
determination of the color, due to the non-negligible time gaps between
observations using two different filters for a particular LSST field. This will
eventually determine the AGN $L-z$ distribution and, in particular, may affect
the identification of quasars at $z\gtsim 6$ if, for example, $Y$-band exposures
are not sufficiently deep.

{\bf Variability:} some AGNs can be effectively distinguished from (variable)
stars, and from quiescent galaxies, by exhibiting certain characteristic
variability patterns (e.g., \citealt{ButlerandBloom2011}). Picking the
right cadence can increase the effectiveness of AGN selection. Ultimately,
hybrid color and variability algorithms will be employed to enhance
the selection process (e.g., \citealt{Petersetal2015}); this may be
particularly important for selecting obscured sources which comprise a
significant fraction of the entire AGN population.

%Non-uniform
%sampling may ``contaminate'' the variability signal of AGN candidates.

{\bf Astrometry:} In cases where selection by color and variability is
insufficient for a reliable identification, AGNs can be further selected
among sources having zero proper motion, within the uncertainties. The
LSST cadence may affect the level of this uncertainty in each band, and certainly the temporal baseline for proper motion measurement, and
may therefore affect the ability to identify (mostly fainter) AGN.
%
Differential chromatic refraction (DCR), making use of the astrometric offset a
source with emission lines has with respect to a source with a featureless
power-law spectrum, can help in the selection of AGNs and in confirming their
photometric redshifts \citep{KaczmarczikEtal2009}. The DCR effect is more
pronounced at higher airmasses. Therefore, it could be advantageous to have at
least one visit, per source, at airmass greater than about 1.4 (though of course
there is a trade-off with the additional extinction, for faint sources). AGN
selection and photometric redshift confirmation may be affected since the LSST
cadence will affect the airmass distribution, in each band, for each AGN
candidate.
%
The deep drilling fields (DDFs) will provide a truth table for determining
the predictive power of the DCR method as a function of the airmass
distribution of the observations.

The most critical measurement for the AGN census is having a reliable
and precise redshift for each source, obtained both from a photometric
and an `astrometric' redshift from DCR.


% --------------------------------------------------------------------

% \subsection{Metrics}
\subsection{Metrics}
\label{sec:\secname:metrics}

% Ideas for Metrics:
% detection - how many can LSST detect based on the luminosity function
% (depends on the depth in each band for single epoch and coadd)
% (how will this change with each DR)? @ohadshemmer

% classification - How many of these will we actually classify as quasars?
% non-simultaneous colors. variability of QSOs (how does depend on
% cadence/baseline/seasonal gaps?)

The following are most important for the AGN census:

1) Determine the mean (averaged across the sky) {\bf uncertainty on astrometric
redshifts derived from DCR} as a function of airmass, image quality, and
limiting magnitude. These uncertainties should be compared to the
corresponding uncertainties on the photomteric redshift.

2) Estimate the {\bf number of quasars at $z>6$ that LSST can discover}
during a single visit, as well as in the entire survey, and verify that
these numbers do not fall short of the original predictions. To first order this
simply requires computing the maximum depth in the $Y$-band (for both
single visits and the coadd), averaged across the sky for the nominal
OpSim, as well as assessing the ability to reject L and T dwarfs via astrometry.

3) Assess the effect of {\bf non-simultaneous colors on AGN selection.}
First, the term color should be clearly defined. Potential definitions
include the difference between the co-adds in two bands for the entire
survey (or at a certain point in time during the survey), the difference
between the mdian magnitude in each band during the survey, or the
difference between observations in two bands that are closely spaced in time.
Next, each source would be represented as an ellipse in color-color space.
The aim is to assess the sizes of the ellipses and how these sizes could be
minimized by perturbing the current cadence.

4) Assess how the sampling affects the selection of AGN by variability (e.g.,
interactions with red-noise power spectrum).

5) Check how overall survey length affects proper motion measurements and consequently AGN selection.

%4) Estimate the number of low-luminosity AGN (LLAGN) that can be
%identified during the entire survey.

% --------------------------------------------------------------------

% \subsection{OpSim Analysis}
\subsection{OpSim Analysis}
\label{sec:\secname:analysis}

% OpSim analysis: how good would the default observing strategy be, at
% the time of writing for this science project?

\begin{figure}
\centering\includegraphics[width=0.9\linewidth]{figs/agn/zgt6_figure_AAS_2013.png}
\caption{Number of quasars at $z>6$ that LSST is expected to discover
based on $Y$-band limiting magnitude in a single epoch (entire survey)
marked by the first (second) dotted line from the left.}
\label{fig:zgt6}
\end{figure}

% LSST Review from Niel Brandt: check for updates needed to this figure, as it is over a decade old. Also, add a plot comparing LSST and WFIRST for high-z AGN selection

For assessing the limitations of DCR on the $L-z$ plane of LSST AGNs,
one needs to obtain from OpSim the current maximal airmasses for each band,
and the associated image quality and limiting magnitude. The current maximal
airmasses, per band, averaged across the sky are: 1.41, 1.50, 1.51, 1.51, 1.51,
and 1.51 for $u, g, r, i, z, Y$, respectively. Need to convolve this with the
seeing in each band to obtain the dependence on airmass and image
quality. This output should be converted into the mean and spread
of the uncertainty on the astrometric redshifts. The best way
to obtain this is to fold the astrometric redshift estimation from
DCR into MAF. Ultimately, one needs to check the implications of higher
airmasses and limiting magnitudes on the ability to obtain more accurate
and precise astrometric redshifts.


For predicting the number of detected $z>6$ quasars
%Compare this magnitude to the
%one required for identifying $\geq1000$ quasars at $z\geq6$.
the
% current enigma\_1189{}
% aquila\_{}1110
\opsimdbname{aquila\_1110}
\OpSim database for the main, i.e., WFD survey, gives a
single-visit $5\sigma$-depth
%old number computed in Bremerton: $Y=22.36$ mag,
%$Y=21.51$ enigma_\_{}1189
{\it Y} $= 21.45$ mag (as compared to {\it Y} $= 21.51$ for the older \opsimdbname{enigma\_1189} run).
For the final co-added $5\sigma$-depth the median magnitude from the \opsimdbname{aquila\_1110}
\OpSim run is
%$Y=24.36$. Enigma\_{}1189
{\it Y}$ = 23.07$ which is more than a magnitude shallower than the older \opsimdbname{enigma\_1189}
depth of {\it Y}$ = 24.36$.
{\bf The lower {\it Y}-band depth may reduce the total number of quasars at $z > 6$ discovered by LSST by
a factor of $\sim 2$
%correspondingly deeper (i.e., better) than enigma\_{}1189
%comparable to
%the original predictions
(see Fig.~\ref{fig:zgt6}).}
% (See the AAS poster from 2013: http://www.lsst.org/sites/default/files/221-RC-247.10-AAS_shemmer.pptx.pdf).



As for general AGN selection, the effects of the sampling on variability selection
should be assessed, and the amplitudes of the uncertainties in color-color space
and how these depend on the cadence should be simulated.
The combination of LSST photometry with that from WFIRST and/or Euclid data should also be considered, both for extending the upper limit in detectable redshift to $\sim10$, but also improving the completeness and purity of the sample at lower redshifts.

% % --------------------------------------------------------------------
%
% \subsection{Discussion}
% \label{sec:\secname:discussion}
%
% Discussion: what risks have been identified? What suggestions could be
% made to improve this science project's figure of merit, and mitigate
% the identified risks?
%
%
% ====================================================================

\subsection{Conclusions}

Here we answer the ten questions posed in
\autoref{sec:intro:evaluation:caseConclusions}:

\begin{description}

\item[Q1:] {\it Does the science case place any constraints on the
tradeoff between the sky coverage and coadded depth? For example, should
the sky coverage be maximized (to $\sim$30,000 deg$^2$, as e.g., in
Pan-STARRS) or the number of detected galaxies (the current baseline
of 18,000 deg$^2$)?}

\item[A1:] The main FoM for AGN science is maximizing the number of AGNs
detected.  Since the co-added depth is already pushing well into the
regime of low-luminosity AGNs, that suggests that area is preferred over
depth.

\item[Q2,3,5:] {\it Does the science case place any constraints on the
tradeoff between uniformity of sampling and frequency of  sampling? Does
the science case place any constraints on the tradeoff between the
single-visit depth and the number of visits (especially in the $u$-band
where longer exposures would minimize the impact of the readout noise)?
Does the science case place any constraints on the fraction of observing
time allocated to each band?}

\item[A2,3,5:] It should be possible to do a MAF analysis to determine
the relative selection completeness and efficiency for OpSim outputs
with different uniformity/frequency of sampling. The same can be said
for constraining the fraction of observing time in each band (where $u$
is the most important for $z\lesssim3$ and $Y$ is most important for $z\gtrsim6$)
and for determining whether nightly
visits should be in the same band or not, and for the trade-off of
single-visit depth and number of visits. However, the AGN census is
unlikely to be the driver in these decisions.

\item[Q4:] {\it Does the science case place any constraints on the
Galactic plane coverage (spatial coverage, temporal sampling, visits per
band)?}

\item[A4:] Given the desire for maximal extragalactic area, added Galactic plane
coverage would be detrimental to AGN science.

\item[Q6:] {\it Does the science case place any constraints on the
cadence for deep drilling fields?}

\item[A6:] Nearly any cadence discussed for the deep drilling fields
will be more than adequate for AGN selection (as opposed to AGN
physics, which will provide constraints); however added depth during
commissioning would enable more robust truth tables.

\item[Q7:] {\it Assuming two visits per night, would the science case
benefit if they are obtained in the same band or not?}

\item[A7:] No preference.

\item[Q8:] {\it Will the case science benefit from a special cadence
prescription during commissioning or early in the survey, such as:
acquiring a full 10-year count of visits for a small area (either in all
the bands or in a  selected set); a greatly enhanced cadence for a small
area?}

\item[A8:] No preference.

\item[Q9:] {\it Does the science case place any constraints on the
sampling of observing conditions (e.g., seeing, dark sky, airmass),
possibly as a function of band, etc.?}

\item[A9:] No.

\item[Q10:] {\it Does the case have science drivers that would require
real-time exposure time optimization to obtain nearly constant
single-visit limiting depth?}

\item[A10:] The census of AGNs is unlikely to be the determining factor
in terms of observing conditions and does not require nearly constant
single-visit depths.

\end{description}

% LSST Review from Niel Brandt: survey must span full 10 years to enable good astrometry / propoer motion measurements to aid in selection. PJM: doesn't fit in the 10 questions, but leaving a note here for the future!


\navigationbar
