% ====================================================================
%+
% SECTION:
%    AGN_BELR.tex
%
% CHAPTER:
%    agn.tex
%
% ELEVATOR PITCH:
%
%-
% ====================================================================

% \section{The Size and Structure of the Broad Emission Line Region}
\subsection{The Size and Structure of the Broad Emission Line Region}\label{sec:AGNBELR}
\def\secname{\chpname:photoRM}\label{sec:\secname}

\credit{ohadshemmer}

LSST may provide estimates of the size and structure of the broad
emission line region (BELR) using the photometric reverberation
mapping (PRM) method. This method enables one to measure the
time-delayed response of the flux in one band to the flux
in another by using cross-correlation techniques on AGN light
curves (e.g.,
%\citet{Chelouche2013}; \citet{CheloucheandZucker2013};
\citealt{CheloucheEtal2014}).



%; \citet{EdelsonEtal2015}; \citet{FausnaughEtal2015}).
The main challenge of PRM is to detect,
with high confidence, the time lag between the variations of a BELR
line-rich band with respect to variations in a line-poor band, given
the relatively small flux contributions ($\sim10$\%)~of BELR lines to each
LSST band. Nevertheless, LSST is expected to deliver BELR line-continuum
time delays in $\sim10^5-10^6$ sources, which is unprecedented when
compared to $\sim60-80$ such measurements conducted, to date, via the
traditional, yet much more expensive (per source) spectroscopic method.
Sources in the DDFs will benefit from the highest
quality PRM time-delay measurements given the factor of $\sim10$ denser
sampling \citep{CheloucheEtal2014}.

% --------------------------------------------------------------------

% \subsection{Target measurements and discoveries}
\subsubsection{Target measurements and discoveries}
\label{sec:\secname:targets}

The PRM measurements will probe the size and structure of the BELR,
in a statistical sense, and may provide improved SMBH mass estimates
for sources that have at least single-epoch spectra. PRM will also be
used to trigger follow-up spectrophotometric monitoring of ``promising''
cases depending on their variability properties. The goal is to obtain
$R_{\rm BELR}$ measures for different BELR lines in certain luminosity
and redshift bins; for example, PRM may provide mean $R_{\rm BELR}$ for
Ly$\alpha$ in quasars at $2.1\ltsim z \ltsim 2.2$ with
$45 \ltsim \log L ({\rm erg~s}^{-1}) \ltsim 46$, or mean $R_{\rm BELR}$
for C~{\sc iv}~$\lambda 1549$ in quasars at $1.6\ltsim z \ltsim 1.7$
with $44 \ltsim \log L ({\rm erg~s}^{-1}) \ltsim 45$.
%\new{Our goal is to understand the population of
%AGN broad line regions, including their geometry. We expect to do this
%via  a model that connects the BH mass, BLR geometry and AGN
%photometric variability properties via a set of scaling relations. A
%simple version of this is could be something like...\newline\newline
%So, our target measurements are of $a$ and $b$, the X parameters.
%Before we derive a metric that quantifies our ability to measure these
%parameters, we can anticipate some of the sensitivity of the
%photometric RM method to observing strategy.}

% LSST Review by Niel Brandt: how to address selection bias towards high line strength systems?

The PRM method is very sensitive to the sampling in each band,
therefore the ability to derive reliable time delays can be affected
significantly by the LSST cadence. The best results will be obtained
by having the most uniform sampling equally for each band.
%
Since the observed line-continuum lags scale with luminosity and redshift,
PRM with the LSST will be limited by the average time gaps between successive
observations in a particular band.
%
Additionally, there is a trade-off between the number of DDFs and the
number of time delays that PRM can obtain \citep{CheloucheEtal2014}.
For example, an increase in the number of DDFs, with similarly dense
sampling in each field, can yield a proportionately larger number of
high-quality time delays, down to somewhat lower luminosities (to the
extent that host-galaxy contamination can be neglected), but at the
expense of far fewer time delays (for relatively high luminosity
sources) in the main survey.

% --------------------------------------------------------------------

% \subsection{Metrics}
\subsubsection{Metrics}
\label{sec:\secname:metrics}

The average and the dispersion in the number of visits, per band, across
the sky for the nominal OpSim (during the entire survey) should be computed.
Since PRM works best for uniform sampling, one should compare the distributions
of the number of visits in each band, averaged across the sky, and identify
ways to minimize any potential differences between these distributions. By
running PRM simulations, one should identify the 1) minimum number of visits
(in any band) that can yield any meaningful BELR-continuum lag estimates, and
2) the largest difference in the number of visits between two different bands
that can yield any meaningful BELR-continuum lag estimates. These simulations
should be repeated by doubling the nominal number of DDFs. Finally, the
%number of meaningful BELR-continuum time delays that can be obtained
uncertainties on $R_{\rm BELR}$ values achieved with the nominal OpSim
should be assessed, and potential perturbations to the cadence should be
pointed out to reduce these uncertainties.

Another metric is the accuracy of the slope $\alpha$, $\Delta \alpha$, in the
$R_{\rm BELR} \propto L^{\alpha}$ relation. Spectrophotometric monitoring
typically yields $\alpha \simeq 0.50 \pm 0.05$.

% % --------------------------------------------------------------------
%
% \subsection{OpSim Analysis}
% \label{sec:\secname:analysis}
%
% OpSim analysis: how good would the default observing strategy be, at
% the time of writing for this science project?
%
%
% % --------------------------------------------------------------------
%
% \subsection{Discussion}
% \label{sec:\secname:discussion}
%
% Discussion: what risks have been identified? What suggestions could be
% made to improve this science project's figure of merit, and mitigate
% the identified risks?
%
%
% ====================================================================
%
% \subsection{Conclusions}
%
% Here we answer the ten questions posed in
% \autoref{sec:intro:evaluation:caseConclusions}:
%
% \begin{description}
%
% \item[Q1:] {\it Does the science case place any constraints on the
% tradeoff between the sky coverage and coadded depth? For example, should
% the sky coverage be maximized (to $\sim$30,000 deg$^2$, as e.g., in
% Pan-STARRS) or the number of detected galaxies (the current baseline
% of 18,000 deg$^2$)?}
%
% \item[A1:] ...
%
% \item[Q2:] {\it Does the science case place any constraints on the
% tradeoff between uniformity of sampling and frequency of  sampling? For
% example, a rolling cadence can provide enhanced sample rates over a part
% of the survey or the entire survey for a designated time at the cost of
% reduced sample rate the rest of the time (while maintaining the nominal
% total visit counts).}
%
% \item[A2:] ...
%
% \item[Q3:] {\it Does the science case place any constraints on the
% tradeoff between the single-visit depth and the number of visits
% (especially in the $u$-band where longer exposures would minimize the
% impact of the readout noise)?}
%
% \item[A3:] ...
%
% \item[Q4:] {\it Does the science case place any constraints on the
% Galactic plane coverage (spatial coverage, temporal sampling, visits per
% band)?}
%
% \item[A4:] ...
%
% \item[Q5:] {\it Does the science case place any constraints on the
% fraction of observing time allocated to each band?}
%
% \item[A5:] ...
%
% \item[Q6:] {\it Does the science case place any constraints on the
% cadence for deep drilling fields?}
%
% \item[A6:] ...
%
% \item[Q7:] {\it Assuming two visits per night, would the science case
% benefit if they are obtained in the same band or not?}
%
% \item[A7:] ...
%
% \item[Q8:] {\it Will the case science benefit from a special cadence
% prescription during commissioning or early in the survey, such as:
% acquiring a full 10-year count of visits for a small area (either in all
% the bands or in a  selected set); a greatly enhanced cadence for a small
% area?}
%
% \item[A8:] ...
%
% \item[Q9:] {\it Does the science case place any constraints on the
% sampling of observing conditions (e.g., seeing, dark sky, airmass),
% possibly as a function of band, etc.?}
%
% \item[A9:] ...
%
% \item[Q10:] {\it Does the case have science drivers that would require
% real-time exposure time optimization to obtain nearly constant
% single-visit limiting depth?}
%
% \item[A10:] ...
%
% \end{description}

% ====================================================================

\navigationbar
