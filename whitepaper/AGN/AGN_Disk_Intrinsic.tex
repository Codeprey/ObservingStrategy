% ====================================================================
%+
% SECTION:
%    AGN_Disk_Intrinsic.tex
%
% CHAPTER:
%    agn.tex
%
% ELEVATOR PITCH:
%-
% ====================================================================

\section{Disc Intrinsic AGN Variability}\label{sec:AGNContinuum}
\def\secname{\chpname:variability}\label{sec:\secname}

\credit{ohadshemmer},
\credit{AstroVPK},
{\it Robert Wagoner}

A variety of AGN variability studies will be enabled by LSST. These are
intended to probe the physical properties of the unresolved inner regions
of the central engine. Relations will be sought between variability amplitude
and timescale vs. $L$, $z$, $\lambda_{\rm eff}$, color, multiwavelength and
spectroscopic properties, when available. For example, LSST AGNs exhibiting excess
variability over that expected from their luminosities will be further scrutinized
as candidates for lensed systems having unresolved images with the excess
(extrinsic) variability being attributed mainly to microlensing.

Measuring time-delayed responses between variations in the continuum flux in one
LSST band to the continuum flux in another, will be one of the main themes of
AGN science in the LSST era (e.g., \citealt{Chelouche2013};
\citealt{CheloucheandZucker2013}; \citealt{EdelsonEtal2015};
\citealt{FausnaughEtal2015}). Such measurements can test accretion disk models
in a robust manner for a considerably larger number of AGNs than is currently
feasible with microlensing (see section~\ref{sec:agn:microlensing}).

Theories of the hierarchical merger of dark matter halos over cosmic time
predict that galaxy-galaxy mergers should result in the formation of a large
number of binary SMBHs. This population is predicted to be a strong, stochastic
contributor to the overall gravitational wave background
\citep{2015arXiv151105564T}. The inspiral of gravitationally bound pairs of
SMBHs formed by a major merger may `stall', reducing the gravitational wave
signal \citep{2014SSRv..183..189C}. Potentially periodic AGN variability,
leading to tentative discoveries of binary SMBHs (e.g.,
\citealt{2015Natur.525..351D}; \citealt{GrahamEtal2015}; \citealt{LiuEtal2015}),
may be feasible for LSST for periods ranging from a few days up to $\sim3$~yr
over the entire survey. The fraction of close SMBHs, tentatively detected by
LSST, may provide strong constraints on the strength of the graviational wave
signal expected from the inspiral.

In the deep-drilling fields (DDFs), the LSST sampling is expected to provide
high-quality power spectral density (PSD) functions for $\approx10^{5} - 10^{6}$
AGNs across wide ranges of $L$, $z$, and $\lambda_{\rm eff}$ down to frequencies of
$\sim1$~d$^{-1}$. These PSDs can enhance AGN selection, and can be used to
constrain the SMBH mass and accretion rate/mode, as well as enable searching for
periodic or quasi-periodic oscillations (QPOs).

%
%Potentially periodic AGN variability, leading to tentative discoveries
%of binary SMBHs (e.g., \citet{GrahamEtal2015}), may also be
%measurable.
%

%Photometric reverberation mapping (PRM), measuring the time-delayed
%response of either the flux of the broad emission line region (BELR)
%lines to the flux of the AGN continuum or between the continuum flux
%in one (longer wavelength) band to the continuum flux in another (band
%with shorter wavelength), will be one of the cornerstones of AGN
%research in the LSST era (e.g., \citet{Chelouche2013};
%\citet{CheloucheandZucker2013}; \citet{CheloucheEtal2014};
%\citet{EdelsonEtal2015}; \citet{FausnaughEtal2015}). For example, LSST
%is expected to deliver BELR line-continuum time delays in
%$\sim10^5-10^6$ sources, which is unprecedented when compared to
%$\sim50-100$ such measurements conducted via the traditional, yet much
%more expensive (per source) spectroscopic method. Sources in the
%deep-drilling fields (DDFs) will benefit from the highest quality PRM
%time-delay measurements given the factor of $\sim10$ denser sampling.

The high-frequency QPO (HFQPO) periods expected from the inner accretion disk
(which provide stable clocks located closer to the horizon as the BH spin increases)
can be estimated from those of the fundamental $g$-mode, which agree with
the observed HFQPO frequencies in stellar-mass BH binaries. Utilizing the
theoretical upper bounds for BH spin and $L/L_{\rm Edd}$, and the lower
bound to the $k$- and bolometric correction $B_n(z)$, one obtains
$\log P({\rm hr}) > 0.4(1-m_n) + \log[(1+z)D_{\rm L}(z)^2]$ for magnitude
$m_n$ in a particular band $n$ and luminosity distance $D_{\rm L}(z)$.
The $k$- and bolometric correction $B_n(z)$ is a decreasing function of BH mass,
but an increasing function of BH spin. The Lyman-alpha forest limits the redshift
range to $z < 2.7$ for $g$-band observations. The HFQPOs will be weaker within longer
wavelength bands.
%The $\sim 80$ visits in the $g$-band proposed for the main survey appears
%insufficient to produce a useful PSD. The expected HFQPO periods are longer than a few hour visit in a DDF.
For instance, for $m_g  <  23$ and the optimal $z =  2.7$, the HFQPO period is $P > 4$~hr.
%
%QPO search will be most relevant for the DDFs,
Searching for HFQPOs in the DDFs will be most effective if the sampling frequency
in those fields for the $u$ and $g$ bands is at least nightly, i.e., $\gtsim3000$
visits, per band, during the entire survey. Given that the period of a typical HFQPO is
related to the SMBH mass by $P({\rm hr}) \approx (1.1-4.0)(1+z)(M/10^{7}M_{\odot})$,
LSST will be sensitive to probing SMBHs with $M < 10^{7}~M_{\odot}$ using HFQPOs.

%In addition, there's a need for an "ultradeep" field, e.g., the MCs, that will
%be monitored, during commissioning phase, with frequencies in the range ~$1$-$10^4$ min. (i.e., from
%minutes to weeks/months).

% --------------------------------------------------------------------

\subsection{Target measurements and discoveries}
\label{sec:\secname:targets}

%We will measure the power spectral density (PSD) of AGN light
%curves across $L$, $z$, and $\lambda_{\rm eff}$. Specifically, we will
%measure the short timescale ($\leq 5$~d) spectral index of the PSD and
%the locations of `features' such as QPOs
%and breaks in the PSD.

In the main survey, standard time-series analysis techniques will be used for
measuring time delays between pairs of continuum bands and for detecting
periodic AGNs. Correlation analyses will search for relations between AGN
variability properties and their basic physical parameters. In the DDFs, such
analyses will enable probing deeper and more frequently, resulting in
higher-quality data that will provide stronger constraints; the only drawback is
the relatively smaller number of sources available at the high-luminosity end.

A key measurement enabled by the DDFs is a high-quality PSD, in six bands,
for the largest number of AGNs to date. These PSDs, which are rich
in diagnostic power, will be used to search for `features' such as QPOs
and breaks, as well as power-law slopes, that can help constrain SMBH masses
and accretion rates. Additionally, the PSDs can serve as selection
tools, to more effectively distinguish AGNs from variable stars, as
well as a basis to propose cadence perturbations to further enhance
AGN selection.

A high-quality PSD, extending to high frequencies (reaching $\sim 1$ min
timescales for stacked PSDs), can effectively distinguish AGNs from other
variable sources, assuming AGN light curves are described by a particular
continuous-time autoregressive moving average model (C-ARMA; \citet{KellyEtal14}),
i.e., C-ARMA(2,1), corresponding to a damped harmonic oscillator.
%
Determination of the parameters that describe the PSD requires light curve
sampling at least as frequent as $\sim1$~d$^{-1}$. Figure~\ref{PSDvsFreq} shows
the frequency dependence of the spectral index of the PSD for one particular AGN,
Zw 229-15, observed with {\em Kepler}. The light curve of this source is
well-described by a C-ARMA(2,1) model. The C-ARMA(2,1) model is a higher order
random walk than the damped random walk (DRW) model of \citet{Kelly09}, which
corresponds to a C-ARMA(1,0) model. Recent variability studies indicate that
the simple C-ARMA(1,0) model is insufficient to model AGN variability because
the spectral index of its PSD is mathematically constrained to be 2
\citep{KellyEtal14,Kasliwal15,Simm15}. Insufficient sampling of an AGN light
curve (e.g., a few times a month), can therefore result in the erroneous conclusion
that a DRW model adequately characterizes the variability.

\begin{figure}
  \begin{subfigure}[t]{0.5\textwidth}
    \centering\includegraphics[width=0.9\linewidth]{figs/agn/AGN_Variability_01.pdf}
    \centering
    \caption{}
    \label{fig:PSDvsFreq}
  \end{subfigure}
  %\medskip
  \begin{subfigure}[t]{0.5\textwidth}
    \centering\includegraphics[width=0.9\linewidth]{figs/agn/PowerOfSDSSK2.jpg}
    \centering
    \caption{}
    \label{fig:SDSSK2Power}
  \end{subfigure}
  \caption{(\subref{fig:PSDvsFreq}) shows the PSD of Zw 229-15 as a function of frequency,
  obtained from {\em Kepler} photometry. The PSD (purple) is the ratio of an even
  polynomial numerator (orange) to an even polynomial denominator (brown).
  This AGN is well-described by a C-ARMA(2,1) model; different powers of frequency
  dominate its PSD at different frequencies depending on the hyperparameters of
  this model. The wide frequency range enables detection of PSD spectral index variations
  ranging between 0 and 4. Clearly, the light curve of this AGN must be sampled on
  timescales {\em shorter} than $1-5$ days in order to observe the $\nu^{-4}$ behavior
  characteristic of a higher order random walk. This is illustrated in (\subref{fig:SDSSK2Power})
  where we see the frequencies and time intervals probed by SDSS, Kepler and a light curve
  constructed by combining the two datasets (SDSS+K2). Each vertical dashed line corresponds to
  a pair of observations seperated by the indicated $\delta t$ (top axis). We plot (for
  illustration), two C-ARMA models with the same overall power - a damped random walk, i.e. a
  C-ARMA(1.0) process, and a damped harmonic oscillator, i.e a C-ARMA(2,1) process.
  It is clear that SDSS (Kepler) cannot probe the highest (lowest) frequencies. However the
  combination of the two can cover the full frequency range. The LSST cadence should be chosen
  to provide similar temporal coverage in the DDFs.}
  \label{PSDvsFreq}
\end{figure}

Accurate recovery of the PSD parameters can be greatly enhanced by increasing the
sampling frequency. To illustrate the effects of the cadence, Figure \ref{CadenceEffect}
shows how the inferred joint distribution of two hyperparameters of the C-ARMA(2,1)
model, the oscillator timescale and the damping ratio, depend on the sampling frequency.
Degrading the sampling frequency from $1/$($30$~min), corresponding to {\em Kepler}
light curves, to $1/$($3$~d), corresponding to the nominal DDF cadence, changes both
the size and the shape of the joint distribution, degrading both the accuracy and
correlation of the inferred hyperparameters.
%
Furthermore, the C-ARMA formalism may enable adjusting the cadence of the DDFs once
the LSST survey begins to determine the sampling pattern in real time.

\begin{figure}
\centering\includegraphics[width=0.9\linewidth]{figs/agn/AGN_Variability_00.png}
\caption{The effect of sampling frequency on hyperparameter estimation (courtesy of
J.~Moreno). Light curves were generated using a C-ARMA(2,1) model using the best-fit
parameters for Zw 229-15, observed with {\em Kepler}, indicated by the red cross in
each panel. The light curve was then down-sampled to simulate the effect of observational
cadence. Constraints on the oscillator period and damping ratio begin to widen noticeably
at 3-day sampling. At 1 week and longer cadence (not shown), one does not recover the
correct model order. This strongly indicates the importance of further study to refine
the cadence requirements for LSST.
}
\label{CadenceEffect}
\end{figure}

%The PRM measurements will probe the size and structure of the
%accretion disk and BELR, in a statistical sense, and may provide
%improved SMBH mass estimates for sources that have at least
%single-epoch spectra. \new{Our goal is to understand the population of
%AGN broad line regions, including their geometry. We expect to do this
%via  a model that connects the BH mass, BLR geometry and AGN
%photometric variability properties via a set of scaling relations. A
%simple version of this is could be something like...\newline\newline
%So, our target measurements are of $a$ and $b$, the X parameters.
%Before we derive a metric that quantifies our ability to measure these
%parameters, we can anticipate some of the sensitivity of the
%photometric RM method to observing strategy.}

%\new{We focus on the PSD function as a way of characterizing AGN
%variability in various ways. What do we expect the AGN population to
%look like in PSD parameter space? The hyper-parameters that govern the
%relationships between PSD parameters and  AGN and host galaxy
%properties are probably of greatest scintific interest.}

% --------------------------------------------------------------------

\subsection{Metrics}
\label{sec:\secname:metrics}

% Quantifying the response via MAF metrics: definition of the metrics,
% and any derived overall figure of merit.

%\new{In lieu of a simulated AGN population, we focus on a few
%particular {\it diagnostic} metrics that capture  our likely ability
%to measure the PSD across the population. These include: the
%uniformity of the sampling pattern in log time lag?}


%Assess the number of meaningful BELR-continuum time delays that can be obtained
%with the nominal OpSim, and point out potential perturbations in the
%cadence to improve the number and quality of such time delays.

While additional work is required for determining the optimal cadence in order
to fully capture AGN accretion physics and to enhance AGN selection, it is clear
that even the nominal DDF sampling (\eg in \opsimdbname{enigma\_1189}) is barely sufficient, and
more frequent sampling would have been ideal. The ability to detect HFQPOs
should also improve by increasing the sampling frequency, the amplitudes of such
features are quite uncertain, as are the (short) duty cycles. Observations,
theory and numerical simulations have only suggested that the fractional
modulation should be small (less than a few percent). Thus it is not obvious how
to choose a metric and observing strategy to maximize it, other than increasing
the sampling frequency to at least nightly samplings in the $g$ and $u$ bands
(i.e., increasing the sampling frequency in the DDFs at least by a factor of
$\sim 3$).

Specific metrics include:

1) LSST can make a significant contribution using
the C-ARMA formalism in the selection of low-luminosity AGN (LLAGN), i.e.,
sources with $L \ltsim 10^{42}$~erg~s$^{-1}$, in the DDFs. Such sources are
likely to be missed by traditional color-variability selection algorithms due to
a strong host contribution. The metric to be developed should assess how the
number of selected LLAGN depends on the sampling frequency in each band.

2) Assessing the standard deviation of the error in recovered time-lag between
bands, $\tau$, using a cross-correlation analysis. The goal is to minimize
$\sigma_{\Delta \tau}$. Additionally, one should assess the worst case estimate of
the time-lag between bands, i.e., minimizing $\max \vert \Delta \tau \vert$.

3) Determining the fractional error on the slopes and features of the PSD.
Assuming that AGN variability is parametrized by a C-ARMA process with
autoregressive roots $\{\rho_{i}:1 \leq i \leq p\}$, the damping timescales and
QPO centers are given by $\tau_{i} = 1/|\Re(\rho_{i})|$ and ${\rm QPO}_{i} =
2\pi/|\Im(\rho_{i})|$, respectively. One needs to investigate how well different
sampling strategies recover each damping timescale and QPO center for a range of
assumed models. The choice of appropriate models can be guided by using
variability data from K2 observations of SDSS quasars.

% % --------------------------------------------------------------------
%
% \subsection{OpSim Analysis}
% \label{sec:\secname:analysis}
%
% OpSim analysis: how good would the default observing strategy be, at
% the time of writing for this science project?
%
%
% % --------------------------------------------------------------------

\subsection{Discussion}
% \subsubsection{Discussion}
\label{sec:\secname:discussion}

% Discussion: what risks have been identified? What suggestions could be
% made to improve this science project's figure of merit, and mitigate
% the identified risks?

Overall, the key requirement is to increase the nominal sampling
frequency in the DDFs by at least a factor of 3, i.e., having at least
3000 visits, per band, during the entire survey. Alternatively, if this
sampling is not feasible for all the DDFs, it would be beneficial to
identify a subset of ``special'' DDFs which would be sampled by this
frequency. Such DDFs would also benefit from being circumpolar, e.g.,
the Magellanic Clouds, enabling a more uniform sampling to produce the
highest quality PSDs.

% ====================================================================
%
\subsection{Conclusions}

Here we answer the ten questions posed in
\autoref{sec:intro:evaluation:caseConclusions}:

\begin{description}

\item[Q1:] {\it Does the science case place any constraints on the
tradeoff between the sky coverage and coadded depth? For example, should
the sky coverage be maximized (to $\sim$30,000 deg$^2$, as e.g., in
Pan-STARRS) or the number of detected galaxies (the current baseline 
of 18,000 deg$^2$)?}

\item[A1:] The disc-intrinsic variability science case places no direct
constraints on the tradeoff between the sky coverage and the coadded depth.

\item[Q2:] {\it Does the science case place any constraints on the
tradeoff between uniformity of sampling and frequency of sampling? For
example, a rolling cadence can provide enhanced sample rates over a part
of the survey or the entire survey for a designated time at the cost of
reduced sample rate the rest of the time (while maintaining the nominal
total visit counts).}

\item[A2:] Preliminary studies of the impact of the survey strategy on
the disc-intrinsic variability science case indicate that
%\begin {enumerate}
%\item
having the longest-possible temporal baseline, i.e., the interval between
the first and last observation, is mandatory. The temporal baseline
\emph{must} be at least 2-3 $\times$ the longest timescale ($\tau$) built
into the light curve \citep{2017A&A...597A.128K}. \citet{2010ApJ...721.1014M}
find that $\max \tau \sim 1000$ d suggesting that survey strategies
that do not visit every field at both the beginning and end of the survey will
be sub-optimal for this science case.
%\item
%having intervals of time during which the survey is performed with
%higher sampling frequency is mandatory.
Furthermore, Fig. \ref{fig:PSDvsFreq} demonstrates
that in order to accurately distinguish between various stochastic models of
AGN variability, the high frequency behavior of the light curve must be
sufficiently explored by utilizing the DDFs.
%\end{enumerate}

%In summary, non-uniform sampling is preferred. Some variant of the rolling-cadences
%strategies that provides a long temporal baseline but also periods of very high
%frequency observing is ideal.

\item[Q3:] {\it Does the science case place any constraints on the
tradeoff between the single-visit depth and the number of visits
(especially in the $u$-band where longer exposures would minimize the
impact of the readout noise)?}

\item[A3:] The science cases would benefit from maximizing the number of visits.

\item[Q4:] {\it Does the science case place any constraints on the
Galactic plane coverage (spatial coverage, temporal sampling, visits per
band)?}

\item[A4:] AGN science cases would benefit from minimizing coverage of
the Galactic plane.

\item[Q5:] {\it Does the science case place any constraints on the
fraction of observing time allocated to each band?}

\item[A5:] In general, the science cases place no constraints on
the fraction of observing time allocated to each band.
For QPOs, however, shorter wavelengths are preferred.

\item[Q6:] {\it Does the science case place any constraints on the
cadence for deep drilling fields?}

\item[A6:] The science cases would benefit from maximizing the sampling
rate in the DDFs to at least one visit per day in each band.

\item[Q7:] {\it Assuming two visits per night, would the science case
benefit if they are obtained in the same band or not?}

\item[A7:] Both continuum-continuum lag studies as well as color-variability studies
of AGN would be strongly benefited by having the two (or more) visits per night be in
\emph{different} bands.

\item[Q8:] {\it Will the case science benefit from a special cadence
prescription during commissioning or early in the survey, such as:
acquiring a full 10-year count of visits for a small area (either in all
the bands or in a  selected set); a greatly enhanced cadence for a small
area?}

\item[A8:] The disc-intrinsic variability science case would be benefited by having a
greatly enhanced cadence for a small commissioning area of the sky as long as this area
is followed up with the normal revisit schedule later during the survey in order to
ensure that the temporal baseline for this region is as long as possible.

\item[Q9:] {\it Does the science case place any constraints on the
sampling of observing conditions (e.g., seeing, dark sky, airmass),
possibly as a function of band, etc.?}

\item[A9:] The disc-intrinsic variability science case places no constraints on
the sampling of observing conditions.

\item[Q10:] {\it Does the case have science drivers that would require
real-time exposure time optimization to obtain nearly constant
single-visit limiting depth?}

\item[A10:] No

\end{description}

% ====================================================================

\navigationbar
