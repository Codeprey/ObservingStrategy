% ====================================================================
%+
% SECTION:
%    MW_SFH.tex
%
% CHAPTER:
%    galaxy.tex
%
% ELEVATOR PITCH:
%
%-
% ====================================================================

\section{Star Formation History of the Milky Way}
\def\secname{MW_SFH}\label{sec:\secname}

\credit{pmmcgehee}

\label{sec:\secname:targets}

% This individual section will need to describe the particular
% discoveries and measurements that are being targeted in this section's
% science case. It will be helpful to think of a ``science case" as a
% ``science project" that the authors {\it actually plan to do}. Then,
% the sections can follow the tried and tested format of an observing
% proposal: a brief description of the investigation, with references,
% followed by a technical feasibility piece. This latter part will need
% to be quantified using the MAF framework, via a set of metrics that
% need to be computed for any given observing strategy to quantify its
% impact on the described science case. Ideally, these metrics would be
% combined in a well-motivated figure of merit. The section can conclude
% with a discussion of any risks that have been identified, and how
% these could be mitigated.

LSST gives the opportunity to survey extensive areas
around star formation regions in the Southern hemisphere. Among
others, it would allow to study the Initial Mass Function down to the
sub-stellar limit across different environments. Young stars are
efficiently identified by their variability.

Section 8.10.2 in the LSST Science Book (p298--299) provides a
  thorough scientific motivation for the characterization of young
  stars through variability, including discussion of the observational
  signatures of the diverse physical phenomena driving observed time
  variability. The general observable is strong, irregular flaring
  across the entire $ugrizy$~bandpass of LSST. Flaring can last from
  minutes to years, and at amplitudes from a few tenths to several
  magnitudes.

% WIC - the following para has been removed since it's already present
% verbatim at the end of Section 8.10.2 in the LSST Science Book.

%LSST will increase the sample size for detailed follow-up observations
%due its ability to survey star formations at large heliocentric
%distances and to detect variability in embedded and highly extincted
%young objects that would otherwise be missed in shallower
%surveys. During its operations LSST will also provide statistics on
%the durations of high states, for at least one important tracer
%population (the shorter-duration EXor variables).

% --------------------------------------------------------------------

\subsection{Target measurements and discoveries}
\label{sec:\secname:targets}

The nature of the variability in young stars changes with evolutionary
status. For the youngest stars still undergoing significant mass
accretion, FU Orionis and related outbursts can occur due to
circumstellar disk instabilities. As the natal environment dissipates
and the accretion rates drops, the stars take on a Classical T Tauri
appearance where the variability is primarily due to changes in the
accretion flow and rotational modulation of hot spots resulting from
accretion shocks on the protostellar photosphere. Also present are the
signature of cool spots arising from strong magnetic fields. This cool
spot rotational modulation is responsible for the variability in the
disk-less, and older, weak-line T Tauri stars.

Of particular interest are the FUor and EXor variables, which are
  named after the prototype objects FU Orionis \citep{hartmann96}
  and EX Lupi \citep{herbig01} respectively, and for which
  only a relatively small number of examples are known. In these
  pre-main sequence objects, eruptive outbursts of up to 6 magnitudes
  have been observed, with high state durations from years to
  decades. In addition to triggering follow-up observations, LSST
  should be able to set the first population constraints on the
  duration of high states, particularly for the short end of the
  timescale distribution for these eruptive variables.


% \new{(WIC: Material already in LSST Science Book removed.)}


%{\bf CTTS and WTTS material goes here.}

%Here are the references cited above:\\
%Hartmann \& Kenyon 1996, ARA\&A, 34, 207 \\
%Herbig et al. 2001, PASP, 113, 1547 \\
%Herbig 1977, ApJ, 217, 693 \\
%Aspin et al. 2009, ApJ, 692L, 67 \\
%Hodapp et al. 1996, ApJ, 468, 861 \\
%McGehee et al. 2004, ApJ, 616, 1058 \\


%Describe the discoveries and measurements you want to make.

%Now, describe their response to the observing strategy. Qualitatively,
%how will the science project be affected by the observing schedule and
%conditions? In broad terms, how would we expect the observing strategy
%to be optimized for this science?


% --------------------------------------------------------------------

\subsection{Metrics}
\label{sec:\secname:metrics}

In order to assess the ability of LSST to 1) identify and 2) classify
Young Stellar Objects we need to quantify the variability timescales
and amplitudes of both Class I/II (stars with disks, including
Classical T Tauris) and Class III (Weak-line T Tauris).  Inclusion of
eruptive variables (FUor/EXor) is appropriate as well.

In brief, Weak-line T-Tauris are quasi-periodic with amplitudes of 0.1
to 0.3 mag and periods 1 to $\sim$15 days, so their variability is
comparable to that of $\gamma$ Dor stars. Given the temporal evolution
of cool spots, a period recovery analysis such as shown for RR Lyrae
stars is likely difficult.  The embedded systems and Classical T
Tauris are irregular variables but have been shown to have distinctive
colors due to extinction and the ultraviolet and blue excess arising
from accretion shocks.

\autoref{table:pseudoForExor} shows a possible Figure of Merit for the
recovery by LSST of the distribution of EXor high-state duration in
outburst.


\begin{table}
\small
\begin{tabular}{c p{12cm}}
& {\it Figure of Merit for recovery of EXor high--state duration distribution}\\
\hline
1.  & Produce ASCII lightcurve for eruptive outburst \\
2.  & Initialise large array to store the maps of fraction detected as a function of duration and amplitude. \\
2.  & for {\it duration T} in range \{min, max\}:  \\
3.  & ~~~~ for {\it amplitude A} in range \{min, max\}: \\
4.  & ~~~~~~~~~~ run {\tt mafContrib/transientAsciiMetric} \\
5.  & ~~~~~~~~~~ store the spatial map of the fraction detected for this (A, T) pair \\
6.  & Initialise master arrays to hold the run of duration distribution measurements.\\
7. & Produce distribution of high--state durations and amplitudes from which the simulations will be drawn. \\
8.  & for {\it iDraw} in range \{1, nDraws\}:\\
9.  & ~~~~ construct model population with input duration distribution \\
10.  & ~~~~ Apply the stored metrics from 2-5 to measure fraction recovered \\
11.  & ~~~~ Characterize the duration distribution for this draw \\
12. & ~~~~ Fill the {\it iDraw}'th entry in the master arrays. \\
13. & {\bf FoM 1:} Compute the median and variance of the upper/lower quintiles. \\
14. & {\bf FoM 2:} Evaluate the bias between recovered and input high-state duration. \\
\hline
\end{tabular}
\caption{\new{Steps for Figure of Merit recovering the distribution
  for the duration of EXor high states. See Section \ref{sec:MW_SFH:targets} }}
\label{table:pseudoForExor}
\end{table}


% --------------------------------------------------------------------

%\subsection{OpSim Analysis}
%\label{sec:\secname:analysis}

%OpSim analysis: how good would the default observing strategy be, at
%the time of writing for this science project?

% --------------------------------------------------------------------

\subsection{Discussion}
\label{sec:\secname:discussion}

Galactic star formation regions are largely found at low Galactic
latitudes or within the Gould Belt structure. As such study of young
stars with LSST is closely tied to other science goals concerning the
Milky Way Disk and is subject to the concerns of both crowded field
photometry and the observing cadence along the Milky Way.

The embedded and Classical T Tauri stars also undergo significant and
rapid color changes due to accretion processes. The ability of LSST to
track these variations in color could be limited by the interval
between filter changes.

% ====================================================================
%
% \subsection{Conclusions}
%
% Here we answer the ten questions posed in
% \autoref{sec:intro:evaluation:caseConclusions}:
%
% \begin{description}
%
% \item[Q1:] {\it Does the science case place any constraints on the
% tradeoff between the sky coverage and coadded depth? For example, should
% the sky coverage be maximized (to $\sim$30,000 deg$^2$, as e.g., in
% Pan-STARRS) or the number of detected galaxies (the current baseline 
% of 18,000 deg$^2$)?}
%
% \item[A1:] ...
%
% \item[Q2:] {\it Does the science case place any constraints on the
% tradeoff between uniformity of sampling and frequency of  sampling? For
% example, a rolling cadence can provide enhanced sample rates over a part
% of the survey or the entire survey for a designated time at the cost of
% reduced sample rate the rest of the time (while maintaining the nominal
% total visit counts).}
%
% \item[A2:] ...
%
% \item[Q3:] {\it Does the science case place any constraints on the
% tradeoff between the single-visit depth and the number of visits
% (especially in the $u$-band where longer exposures would minimize the
% impact of the readout noise)?}
%
% \item[A3:] ...
%
% \item[Q4:] {\it Does the science case place any constraints on the
% Galactic plane coverage (spatial coverage, temporal sampling, visits per
% band)?}
%
% \item[A4:] ...
%
% \item[Q5:] {\it Does the science case place any constraints on the
% fraction of observing time allocated to each band?}
%
% \item[A5:] ...
%
% \item[Q6:] {\it Does the science case place any constraints on the
% cadence for deep drilling fields?}
%
% \item[A6:] ...
%
% \item[Q7:] {\it Assuming two visits per night, would the science case
% benefit if they are obtained in the same band or not?}
%
% \item[A7:] ...
%
% \item[Q8:] {\it Will the case science benefit from a special cadence
% prescription during commissioning or early in the survey, such as:
% acquiring a full 10-year count of visits for a small area (either in all
% the bands or in a  selected set); a greatly enhanced cadence for a small
% area?}
%
% \item[A8:] ...
%
% \item[Q9:] {\it Does the science case place any constraints on the
% sampling of observing conditions (e.g., seeing, dark sky, airmass),
% possibly as a function of band, etc.?}
%
% \item[A9:] ...
%
% \item[Q10:] {\it Does the case have science drivers that would require
% real-time exposure time optimization to obtain nearly constant
% single-visit limiting depth?}
%
% \item[A10:] ...
%
% \end{description}

% ====================================================================

\navigationbar
