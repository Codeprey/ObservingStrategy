\setcounter{chapter}{0}
\chapter*{Preface}
\def\chpname{preface}\label{chp:\chpname}
\addcontentsline{toc}{section}{Preface}
\markboth{}{}

\noindent The Large Synoptic Survey Telescope (LSST) is a dedicated
ground-based astronomical facility whose goal is to provide a database
of high fidelity images and object catalogs that enable a wide range of
science investigations. With its 9.6~square degree field of view and
effective aperture of 6.7~meters, it will be able to survey the Southern half of the sky every few nights (on average), building up a 10-year, 900-frame movie of the ever-changing cosmos. Its community of scientists will be able to make major contributions in the fields of extragalactic astronomy and cosmology, the study of our Milky Way, its local environment and its stellar populations, solar system science, and time domain astronomy.

\noindent As its name suggests, LSST is designed to
carry out a large synoptic survey: it has a baseline observing strategy, simulations of which demonstrate that the data required for
the promised science can be delivered.
However, this baseline strategy may well not be the {\it
best} way to schedule the telescope. Smaller, specialized surveys are
likely to provide high scientific value, as is optimizing the pattern of
repeated sky coverage.  The baseline strategy is not set in
stone, and can and will be optimized: even small changes could result in
significant improvements to the overall science yield. How can we design
an observing strategy that maximizes the scientific output of the LSST
system?

\noindent The LSST Observing Strategy community formed in July 2015 to
tackle this problem. Drawn primarily (but not exclusively) from the team of people engaged in the LSST construction Project, and the set of LSST ``Science Collaborations'' who are engaged in preparing to exploit the LSST data, we are working together
to use software tools provided by the LSST Project
to evaluate simulations of the LSST survey (also provided by the
Project) specifically for the science that we each care most about. In
this way, we aim to give sustainable, quantitative feedback about how any
proposed observing strategy would impact the performance of our science
cases, and so enable good decisions to be made when the telescope
schedule is eventually set up.

\noindent This white paper is a compendium of ideas and results
generated by the community, assembled so that everyone can follow along
with the analysis. It is a living document, whose purpose is to bind
together the group of people who are thinking about the LSST observing
strategy problem, and facilitate their collective discussion and
understanding of that problem (a process we might think of as  ``cadence
diplomacy''). Its audience is the LSST science community, and most notable its Science
Advisory Committee and Project Scientist who together will in the end decide what the LSST observing strategy will be. This white paper is {\it
the} vehicle for the community to communicate to the LSST Project, while
the baseline observing strategy continues to be improved.

\noindent The white paper's modular design allows pieces of it to be
split off and published in a series of snapshot journal papers, as the
various metric analyses reach maturity. The white paper itself will be
continuously published on
\href{https://github.com/LSSTScienceCollaborations/ObservingStrategy}{\GitHub}
and advertized periodically on \href{http://arxiv.org}{astro-ph}. This
white paper is large, but we hope that its hyperlinked structure helps
our community quickly find the science cases that they are most interested in,
starting from the \hyperref[toc]{table of contents}.

\noindent The LSST observing strategy evaluation and optimization
process will be as open and inclusive as possible. New community members
are welcome at any time; we explain how to get involved in \autoref{chp:intro}.  We invite all stakeholders to participate.

\vspace{2\baselineskip}

{\raggedleft \credit{drphilmarshall}, \credit{ivezic} and \credit{bethwillman} \\
 \medskip \hspace{0.8\linewidth} \it June 6, 2017.}

\clearpage
