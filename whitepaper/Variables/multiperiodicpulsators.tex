% ====================================================================
%+
% NAME:
%    multiperiodicvariables.tex
%
% CHAPTER:
%    variables.tex
%
% ELEVATOR PITCH:
%-
% ====================================================================

\section{Characterizing Multiperiodic, Short-Period Pulsating Variables}
\def\secname{multiperiodicvariables}\label{sec:\secname}

\credit{keatonb}

Most pulsating variable stars exhibit a superposition of multiple
simultaneous pulsation modes.  While these multiperiodic pulsators are
observationally more complex than classical pulsators, they communicate
a greater wealth of information about the stars.  Global nonradial
pulsations pass through and are affected by the interiors of stars, and
the measured frequencies of photometric variability are eigenfrequencies
of stars as physical systems.  Therefore, the observation and study of
photometric variability in multiperiodic pulsating stars is the most
powerful method by which we can probe stellar interiors.

The current state and modern tools of the field of asteroseismology are
thoroughly discussed by \citet{2010aste.book.....A}.  The locations of
the known classes of pulsating variable in the H-R Diagram are indicated
in their Figure 1.12, and a summary of their general
properties---including pulsation amplitudes and timescales---is provided
in their Appendix A.

Sections 8.6 and 8.7 of the Science Book express the anticipated
contributions that LSST will make to the study of pulsating variables.
While the sparseness of observations (low duty cycle) over the LSST
survey lifetime will make exact period solutions difficult, if not impossible for
most multiperiodic, short-period pulsating variables, the Science Book
correctly emphasizes that the photometric precision and multi-color
information will yield detections of variability that can be treated
statistically to determine the ensemble properties of different classes
of pulsating star.  The dependence of pulsational power on a star's
position in 5-color parameter space, as well as the relative pulsation
amplitudes in different passbands, informs the least understood aspects
of pulsation theory: mode selection and amplitude limiting mechanisms. A
thorough statistical view of pulsating stars requires on the order of
thousands of objects per type at a minimum, with robust detections of
variability in multiple passbands.

\subsection{The Case for ZZ Cetis}
\label{sec:\secname:targets}

Of the known classes of pulsating variable star, the ZZ Cetis
(pulsating hydrogen-atmosphere white dwarfs) are the faintest, and among
those with the lowest pulsation amplitudes.  These will be the most
difficult for LSST to detect and characterize, and they therefore
provide an important benchmark for assessing the effectiveness of
different proposed observing strategies for the study of pulsating
stars.  If LSST is well suited for ZZ Ceti science, it will be a useful
tool for all classes of pulsator.

Most details of ZZ Ceti stars are not strictly relevant to this
analysis, so we direct the interested reader to recent reviews by
\citet{2008ARA&A..46..157W}, \citet{2008PASP..120.1043F}, and
\citet{2010A&ARv..18..471A}.

As white dwarfs with atmospheres spectroscopically dominated by hydrogen
cool to between roughly 12,500 and 10,600\,K, they are observed as the
photometrically variable ZZ Cetis.  The square root of the total
observed pulsational power (intrinsic root-mean-squared signal) in these
objects is on the order of 1\% and the mean pulsation periods are
$\sim$10\,min \citep[e.g.,][]{2006ApJ...640..956M}.  Because the pulsation
periods are $\ll$ the typical LSST revisit time, the survey will essentially
sample the pulsations randomly in phase.

The ability to detect pulsations relies on the recognition that scatter
in the flux measurements significantly exceeds what can be attributed to
noise.  LSST's sensitivity to these pulsations depends primarily on two things:
the photometric precision and the total number of measurements.  By
affecting these survey characteristics, the choice of observing strategy
impacts LSST's success in its goal of exploring the variable universe.
Strategies that maximize photometric precision and the total number of
visits in all filters optimize the survey toward this goal, but there are
tradeoffs between these dual requirements related primarily to exposure
time that are complicated and must be explored in MAF to be understood.

While consideration of observations across all filters together provides
the greatest sensitivity to detecting overall variability, the measurement
of pulsational power in individual filters serves the science needs for
pulsating stars best.  Since sparse temporal sampling will make measuring
the individual periods of complex, multi-modal pulsating stars challenging,
LSST's greatest contributions to this field may lie
in its ability to measure relative pulsational power across many
passbands.  For ZZ Cetis in particular, there is a dependence of
the relative amplitudes measured in different filters on the geometry of
the pulsations---specifically the spherical degrees, $\ell$, of the
spherical harmonic wave patterns associated with the pulsation modes.
Determining the $\ell$ values associated with individual modes is
essential for comparing measured pulsation frequencies with those
calculated for asteroseismic stellar models.  The difficulty in
determining $\ell$ is currently the greatest limitation on white dwarf
asteroseismology. LSST has the potential to statistically constrain the
relative contributions of modes of different $\ell$ to the
overall photometric variations.  The calculations by
\citet{1995ApJS...96..545B} of relative pulsation amplitudes in
different filters show that measuring the amplitude in the $u$ band is
crucial for gaining leverage on this problem.

% --------------------------------------------------------------------

\subsection{Metrics}
\label{sec:\secname:metrics}

We have developed a custom MAF metric that calculates a ``variability
depth'' for every point on the sky equal to the magnitude limit for
detecting a population of photometric variables with a given
disk-integrated root-mean-squared (r.m.s.)\ underlying signal to a
desired level of completeness and a tolerable level of contamination.
The metric makes the simplifying
assumptions that the typical revisit time for a field is longer than the
pulsation periods (appropriate for many pulsators, including ZZ Cetis)
and that the intrinsic variability takes the form of a Gaussian (which,
for multi-periodic pulsators, is supported by the central limit
theorem).  The metric relies on the total number of visits and
signal-to-noise per visit (scaled from the 5$\sigma$-depth, with
Gaussian errors assumed) for the calculation, and is included in {\tt
sims\_maf\_contrib} as {\tt VarDepth}.  Example output of this metric
is displayed in Figure~\ref{fig:vardepth}.

\begin{figure}
  \centering
  \includegraphics[width=0.76\columnwidth]{figs/vardepth.png}
  \caption{Example output of the {\tt VarDepth} MAF metric run on the
  current baseline cadence, \opsimdbref{db:baseCadence}, after 10 years of survey
  operations. Input parameters and SQL queries were set to calculate the
  magnitude limit for detecting 90\% of pulsators with 1\% r.m.s.\
  variability from a cut on the measured variance in the $r$ band
  (allowing contamination from 10\% of nonvariable sources).}
  \label{fig:vardepth}
\end{figure}


We specialize this metric for the case of ZZ Ceti pulsators by including
maps of the expected distribution of ZZ Cetis in the Galaxy. These maps
were precomputed for the $u$ and $r$ filters by querying the CatSim
database for white dwarfs with an SQL constraint to include only those
with hydrogen atmospheres and with effective temperatures between 10,600
and 12,500 K (i.e., inside the ZZ Ceti instability strip).  While the
boundaries of the instability strip depend slightly on surface gravity,
the width of the strip does not change much, so these temperature cuts
yield representative counts.  The white dwarf spectral energy
distributions were calculated for CatSim by Pierre Bergeron et
al.\footnote{\url{http://www.astro.umontreal.ca/~bergeron/CoolingModels/}}\
The {\tt ZZCetiCounts} metric calculates the number of ZZ Cetis that we
expect to detect in each part of the sky, and the sum of the results
gives the total number of ZZ Cetis with variability detected by LSST.

The measured r.m.s.\ scatter from pulsations in ZZ Cetis is typically of
order $\sim$1\%, with a mean value around 3\%
\citep{2006ApJ...640..956M}.  Since we aim to statistically determine
the ensemble amplitude properties in multiple filters for a large
population of ZZ Cetis (particularly in the $u$ filter), we adopt the
following figure of merit for LSST's ability to study pulsating stars:
that significant variability should be detected in the $u$ band for at
least 1000 ZZ Cetis by the end of
the 10\,yr survey operations.  A sample of this size can be binned to track
changes in pulsational power as white dwarfs cool across the ZZ Ceti
instability strip, without the results being unduly influenced by random
sampling of inclination angle or the white dwarf mass distribution. It also
ensures that LSST contributes an order-of-magnitude improvement to the
number of ZZ Cetis known.

% --------------------------------------------------------------------

\subsection{\OpSim Analysis}
\label{sec:\secname:analysis}

For comparison purposes, we calculate the number of ZZ Cetis detected in
both the $u$ and $r$ filters, assuming intrinsic r.m.s.\ variability of
1\% for two of the currently available \OpSim runs:
\opsimdbref{db:baseCadence}, the current baseline cadence, and
\opsimdbref{db:DoubleUbandExptimeSameVisits}, with doubled $u$-band exposure
times. We require that 90\% of ZZ Cetis with 1\% r.m.s.\  variability
are detected to the computed ``variability depth,'' with a tolerance for
up to 10\% of nonvariables with the same $u$ and $r$ magnitudes to yield
false detections.  The total number of ZZ Cetis detected for each of
these analyses, \emph{if all ZZ Cetis exhibit exactly the assumed level
of r.m.s.\ variability}, is provided in \autoref{tab:zz1pertab}.


\begin{table}[h]
\begin{center}
    \caption{ZZ Ceti Recovery for 1\% R.M.S.\ Variability}\label{tab:zz1pertab}
    \begin{tabular}{| l | l | l |}
    \hline
    \OpSim Run & Filter & \# ZZ Cetis \\ \hline
      \opsimdbref{db:baseCadence} & $u$ & 9  \\
      & $r$ & 127 \\ \hline
      \opsimdbref{db:DoubleUbandExptimeSameVisits} & $u$ & 17\\
      & $r$ & 123  \\ \hline
    \end{tabular}
\end{center}
\end{table}

Clearly LSST appears to fall very short of the proposed figure of merit
by this measure.   However, the 1\% level of variability assumed for ZZ
Cetis in this treatment was chosen to assess how well the
lowest-amplitude ZZ Cetis are recovered by LSST and doesn't fairly
represent the more typical, higher amplitude variables.  If we relax
this constraint and repeat the analysis with ZZ Cetis all modeled as 3\%
r.m.s.\ variables (the mean r.m.s.\ variation observed), we get the
results shown in \autoref{tab:zz3pertab}. This better represents a
total number of ZZ Cetis expected, allowing for incompleteness to low
amplitude.  While we still do not detect $\sim$1000 ZZ Cetis in $u$,
overall we see an increase in the total number of detected ZZ Cetis by
roughly an order of magnitude (in $r$), with the
\opsimdbref{db:DoubleUbandExptimeSameVisits} simulation reaching markedly closer
to the $u$-band goal than the \opsimdbref{db:baseCadence} strategy.


\begin{table}[h]
\begin{center}
    \caption{ZZ Ceti Recovery for 3\% R.M.S.\ Variability}\label{tab:zz3pertab}
    \begin{tabular}{| l | l | l |}
    \hline
    \OpSim Run & Filter & \# ZZ Cetis \\ \hline
     \opsimdbref{db:baseCadence} & $u$ & 197  \\
      & $r$ & 1601 \\ \hline
     \opsimdbref{db:DoubleUbandExptimeSameVisits}  & $u$ & 325\\
    & $r$ & 1534  \\ \hline
    \end{tabular}
\end{center}
\end{table}


% --------------------------------------------------------------------

\subsection{Discussion}
\label{sec:\secname:discussion}

This simplistic analysis ignores most subtleties of ZZ Ceti variables,
but Table~\ref{tab:zz3pertab} captures the order of magnitude of
expected ZZ Ceti detections \emph{from excess scatter alone} for
the \OpSim runs considered.  More sophisticated analysis
of the LSST data will recover additional variables, especially since the
information about the timing of visits has been completely ignored here.
The statistical treatment of large populations of stars will also improve the
detection of variability over this star-by-star treatment.
Still, it appears that the science results for the lowest amplitude ZZ Cetis
will be severely limited, and therefore LSST will capture a less-than-complete
census of pulsating variable stars.

We emphasize again that low-amplitude ZZ Cetis are the most difficult
pulsating stars to observe. LSST will capture variability in higher
amplitude ZZ Cetis and other types of variable star more completely, and
LSST observations of all candidate pulsating white dwarfs will certainly
still be worthy of careful analysis.  We note that the
\opsimdbref{db:DoubleUbandExptimeSameVisits} strategy with 60\,s exposures in $u$
does a much better job of measuring stellar variability overall, at the
expense of a practically negligible decrease in $r$-band detections.  We
argue that any observing strategy that increases the typical {\tt VarDepth}
limit in the $u$ band will improve LSST's scientific yield for
\emph{all} stellar pulsation studies.


% ====================================================================

 \subsection{Conclusions}

 Here we answer the ten questions posed in
 \autoref{sec:intro:evaluation:caseConclusions}:

 \begin{description}

 \item[Q1:] {\it Does the science case place any constraints on the
 tradeoff between the sky coverage and coadded depth? For example, should
 the sky coverage be maximized (to $\sim$30,000 deg$^2$, as e.g., in
 Pan-STARRS) or the number of detected galaxies (the current baseline 
 of 18,000 deg$^2$)?}

 \item[A1:] Coadded depth is not directly a limitation on detecting variability,
 but both improve with the number of revisits to the same field, $N$,
 as roughly $\sqrt{N}$.  A larger field of view also increases the potential population
 of variable stars discovered, but pushing to higher airmass where data quality is
 poorer does not make for a good trade unless the survey area is extended to strategically
 target more variable stars, e.g., over more of the Galactic plane.

 \item[Q2:] {\it Does the science case place any constraints on the
 tradeoff between uniformity of sampling and frequency of  sampling? For
 example, a rolling cadence can provide enhanced sample rates over a part
 of the survey or the entire survey for a designated time at the cost of
 reduced sample rate the rest of the time (while maintaining the nominal
 total visit counts).}

 \item[A2:] While the time distribution of observations has not been considered
 here, this will significantly impact variable star science with LSST.
 Future development of MAF metrics should address specifically how
 the distribution of visits (rather than only number, area, and signal-to-noise as addressed
 above) ease the challenges of amplitude and \emph{period} recovery in sparely sampled
 observations.  Revisit times much shorter than pulsation timescales ($< \sim$5\,min)
 can help to constrain the periods.  Perhaps more importantly, cadences that
 produce the least complicated spectral windows and the highest
 ``effective Nyquist frequencies'' will best facilitate period recovery \citep[e.g.,][]{1999A&AS..135....1E}.

 \item[Q3:] {\it Does the science case place any constraints on the
 tradeoff between the single-visit depth and the number of visits
 (especially in the $u$-band where longer exposures would minimize the
 impact of the readout noise)?}

 \item[A3:] We specifically support increased time spent in the
 $u$-band, with the goal of measuring the relative pulsation amplitudes
 in multiple filters. Increased signal-to-noise per $u$-band exposure enables
 more measurements of pulsational power in $u$, though pulsating star
 science would be hurt by a corresponding decrease in the total number of
 $u$-band visits per field .

 \item[Q4:] {\it Does the science case place any constraints on the
 Galactic plane coverage (spatial coverage, temporal sampling, visits per
 band)?}

 \item[A4:] The Galactic plane contains a higher concentration of multi-periodic
 pulsating stars that LSST will be able to characterize in the time domain.
 Pulsating star science benefits from more survey time spent in the Galactic
 plane.

 \item[Q5:] {\it Does the science case place any constraints on the
 fraction of observing time allocated to each band?}

 \item[A5:] For the ZZ Ceti case study, $u$-band observations are particularly
 important for potentially constraining pulsation geometries. The metric
 developed in this section measures the expected number of ZZ Cetis
 detected in the $u$-band, and we advocate for observing strategies
 that increase this number.

 For classical pulsators, such as $\delta$ Scuti stars (P$\sim$hrs) and $\gamma$ Dor stars (P$\sim$days), multi-band photometry is important for mode identification. Enough data in each band is required to determine the pulsation's amplitude and phase. For these pulsators, the $u$-band is the least informative \citep[e.g.,][]{1990A&A...234..262G}.

 \item[Q6:] {\it Does the science case place any constraints on the
 cadence for deep drilling fields?}

 \item[A6:] The results of the {\tt VarDepth} MAF metric run on the
 current baseline cadence, \opsimdbref{db:baseCadence}, shows that after
 10 years, deep drilling fields are sensitive to pulsation
 \emph{amplitude} measurements in the $r$-band $\approx$1.5\,mag fainter
 than for identical objects in the main survey.  The improvement to
 period recoverability could be far greater in deep drilling fields, but
 new MAF metrics must be developed to demonstrate this and to help
 select between specific proposed cadences.

 The possibility to obtain data at a high cadence for extended periods of time in the deep drilling fields will also facilitate the asteroseismology of solar-like oscillators ($\sim$minutes for main-sequence stars and $\sim$hours for red giants). However, a MAF metric is required to determine the specific requirements.

 \item[Q7:] {\it Assuming two visits per night, would the science case
 benefit if they are obtained in the same band or not?}

 \item[A7:] Analyses that will successfully extract the maximum pulsation
 science from LSST will have to address data across all passbands together,
 and there is no current indication of how or if the nightly revisit filter selection
 will affect this.

 \item[Q8:] {\it Will the case science benefit from a special cadence
 prescription during commissioning or early in the survey, such as:
 acquiring a full 10-year count of visits for a small area (either in all
 the bands or in a  selected set); a greatly enhanced cadence for a small
 area?}

 \item[A8:] A few hours of continuous sampling of a few fields during
 commissioning would produce immediate science results for many
 multi-modal pulsating stars, as well as help accelerate LSST's ultimate
 characterization of variable stars in these fields.

 Furthermore, by mimicking the 10-yr survey for a small area, the scientific community can obtain evidence-based estimates of the yield of different variable classes and example power spectra. This will also provide proof-of-concept data that will facilitate the recruitment of scientists to work on LSST.


 \item[Q9:] {\it Does the science case place any constraints on the
 sampling of observing conditions (e.g., seeing, dark sky, airmass),
 possibly as a function of band, etc.?}

 \item[A9:] The {\tt VarDepth} MAF metric can help assess how the
 adjustments made by the observing strategy to observing conditions
 affect the variable star science output.


 \item[Q10:] {\it Does the case have science drivers that would require
 real-time exposure time optimization to obtain nearly constant
 single-visit limiting depth?}

 \item[A10:] No. Consistent exposure times are simpler to interpret as they
 smooth over any underlying variability by a consistent amount.

 \end{description}

% ====================================================================

\navigationbar
