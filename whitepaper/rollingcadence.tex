% ====================================================================
%+
% NAME:
%    rollingcadence.tex
%
% CHAPTER:
%    cadexp2.tex
%
% ELEVATOR PITCH:
%
% AUTHORS:
%    Steve Ridgway (@StephenRidgway)
%-
% ====================================================================

\section{Future Work: Rolling Cadence}
\def\secname{rolling}\label{sec:\secname}

\credit{StephenRidgway}

With a total of $\sim 800$ visits spaced approximately uniformly over 10
years, and distributed among 6 filters, it is not clear that LSST can
offer the sufficiently dense sampling in time for study of transients
with typical durations less than or $\simeq 1$week. This is particularly
a concern for key science requiring well-sampled SNIa light curves.
``Rolling'' cadences stand out as a general solution that can
potentially enhance sampling rates by 2$\times$ or more, on some of the
sky all of the time and all of the sky some of the time, while
maintaining a sufficient uniformity for survey objectives that require
it. In this section we provide an introduction to the concept of rolling
cadence, and give some examples of ways in which it can be implemented.
% We also present three preliminary \OpSim experiments, that serve to
% illustrate some of the features and challenges of the rolling cadence
% strategy.

% --------------------------------------------------------------------

\subsection{The Uniform Cadence}

Current schedule simulations allocate visits as pairs separated by 30-60
minutes, for the purposes of identifying asteroids.  For most other science
purposes, the 30-60 minute spacing is too small to reveal temporal
information, and a pair will constitute effectively a single epoch of
measurement.  If the expected 824 (design value) LSST visits are
realized as 412 pairs, and distributed uniformly over 10 observing
seasons of 6 months each, the typical separation between epochs will be
4 days (typically in different filters).   The most numerous visits will be in the $r$ and $i$
filters, and the repeat visit rate in either of these will be $\simeq$
20 days.

The possibility is still open that, for asteroid identification, visits
might be required as triples or quadruples (which provide a more robust tracking), in which case the universal
temporal sampling will be further slowed by 1.5 or 2$\times$.

Under a strict universal cadence it is not possible to satisfy a need
for more frequent sample epochs.  This has led the LSST Project
simulations group to investigate the options opened up by reinterpreting
the concept of a universal cadence.  Instead of aiming for a strategy
which attempts to observe all fields ``equally'' all the time, it would
allow significant deviations from equal coverage during the survey,
returning to balance at the end of the survey.

Stronger divergence from a universal cadence, allowing significant
inhomogeneities to remain at the end of the survey, is of course
possible, but is not under investigation or discussed here.

There is currently considerable interest in the community in strategies
that provide enhanced sampling over a selected area of the sky, and
rotating the selected area in order to exercise enhanced sampling over
all of the survey area part of the time.  The class of cadences that
provides such intervals of enhanced visits, with the focus region
shifting from time to time, is termed here a ``rolling cadence.''  As a
point of terminology, observing a single sky area with enhanced cadence
for a period of time will be described as a ``roll''.

% --------------------------------------------------------------------

\subsection{Rolling Cadence Basics}

Assume a fixed number of observing epochs for each point on the sky,
nominally distributed uniformly over the 10-year survey duration.  A subset of
these can be reallocated to provide improved sampling of a given sky region in a given time interval.
This will have the inevitable effects of: (1) reducing the number of
epochs available for that sky region during the rest of the survey (affecting, for example, proper motion studies), and
(2) displace observations of other sky regions during the time of the
improved temporal sampling (affecting, for example, early large scale structure studies needing high depth uniformity).  In short, the cadence outside the enhanced
interval will be degraded.

The essential parameters of rolling cadence are: (1) the number of
samples taken from the uniform cadence, and (2) the enhancement factor
for the observing rate.
\href{https://project.lsst.org/meetings/ocw/sites/lsst.org.meetings.ocw/files/OpSim%20Rolling%20Cadence%20Stratgey-ver1.3.pdf}{LSST document 16370},
``A Rolling Cadence Strategy for the Operations Simulator'', by K. Cook
and S. Ridgway, contains more detailed discussion and analysis.

% --------------------------------------------------------------------
%
% \subsection{Some Example Rolling Cadence \OpSim Runs}
%
% In this section we present a set of three experimental ``Swiss Cheese''
% rolling cadence strategies, and a uniform strategy for use as a control.
%
% %%%%%%%%%%%%%%%%%%%%%%%%%%%%%%%%%%%%%%%%%%%%%%
% \opsimdb[db:SwissCheeseRollingCadence1]{ops2\_1102}{``Swiss Cheese'' rolling cadence, version 1.}
% %%%%%%%%%%%%%%%%%%%%%%%%%%%%%%%%%%%%%%%%%%%%%%%
%
% % {\bf Motivation and description:}
%
%
% %%%%%%%%%%%%%%%%%%%%%%%%%%%%%%%%%%%%%%%%%%%%%%
% \opsimdb[db:SwissCheeseRollingCadence2]{enigma\_1260}{``Swiss Cheese'' rolling cadence, version 2.}
% %%%%%%%%%%%%%%%%%%%%%%%%%%%%%%%%%%%%%%%%%%%%%%%
%
% % {\bf Motivation and description:}
%
%
% %%%%%%%%%%%%%%%%%%%%%%%%%%%%%%%%%%%%%%%%%%%%%%
% \opsimdb[db:SwissCheeseRollingCadence3]{enigma\_1261}{``Swiss Cheese'' rolling cadence, version 3.}
% %%%%%%%%%%%%%%%%%%%%%%%%%%%%%%%%%%%%%%%%%%%%%%%
%
% % {\bf Motivation and description:}
%
%
% %%%%%%%%%%%%%%%%%%%%%%%%%%%%%%%%%%%%%%%%%%%%%%
% \opsimdb[db:RollingCadenceControl]{ops2\_1098}{``Control'' simulation for comparison with the ``Swiss Cheese'' experimental rolling cadence simulations}
% %%%%%%%%%%%%%%%%%%%%%%%%%%%%%%%%%%%%%%%%%%%%%%%
%
% % {\bf Motivation and description:}
%
%
%
% % % - - - - - - - - - - - - - - - - - - - - - - - - - - - - - - - - - - -
%
% \autoref{fig:rollingcadence} illustrates the difference in how survey
% depth (characterized by the number of visits) builds up between uniform
% cadence (\opsimdbref{db:RollingCadenceControl}) and rolling cadence
% (\opsimdbref{db:SwissCheeseRollingCadence2}) strategies.
%
% \begin{figure}
%   \includegraphics[width=2.3in]{figs/ops2_1098_Count_expMJD_r_and_night_lt_365_HEAL_SkyMap.pdf}\includegraphics[width=2.3in]{figs/ops2_1098_Count_expMJD_r_and_night_lt_730_HEAL_SkyMap.pdf}\includegraphics[width=2.3in]{figs/ops2_1098_Count_expMJD_r_and_night_lt_3653_HEAL_SkyMap.pdf} \\
%   \includegraphics[width=2.3in]{figs/enigma_1260_Count_expMJD_r_and_night_lt_365_HEAL_SkyMap.pdf}\includegraphics[width=2.3in]{figs/enigma_1260_Count_expMJD_r_and_night_lt_730_HEAL_SkyMap.pdf}\includegraphics[width=2.3in]{figs/enigma_1260_Count_expMJD_r_and_night_lt_3653_HEAL_SkyMap.pdf}
%   \caption{Example of a regular uniform survey
%   (\opsimdbref{db:RollingCadenceControl}, top row) and a rolling
%   cadence survey (\opsimdbref{db:SwissCheeseRollingCadence2}, bottom row)
%   after 1, 2, and 10 years in the $r$ filter.
%   For the regular survey, the number of visits for any part of the sky
%   is relatively constant throughout the survey.  For the rolling cadence
%   simulation, there are regions with many more exposures in year one
%   which then fade in year two as other parts of the sky are
%   emphasized.}
%   \label{fig:rollingcadence}
% \end{figure}
%
%
% --------------------------------------------------------------------

\subsection{Supernovae and Rolling Cadence}
\label{sec:rolling:supernovae}

% Supernovae as a science topic are addressed elsewhere.
% In this section, the demands of SN are used to directly constrain or
% orient the rolling cadence development.

Pending more quantitative guidance, the SN objective for rolling cadence
is to obtain multi-band time series that are significantly longer than
the typical SN duration, and that have a cadence significantly faster
than uniform. As an example we discuss the option of a rolling cadence
with the regular distribution of filters.

As a simple example, consider improving the cadence by a factor of 2 or
3.  If we accept that some regions of the sky will be enhanced every
year, and that uniform sky coverage will only be achieved at the end of 10
years, then we could use, e.g., 10\% of the total epochs in a single
roll.  If the enhancement is 2$\times$, each roll would last for
$\simeq$ 6 months, with high efficiency for capture of complete SN
events.  If the enhancement is 4$\times$, each roll would last for 2
months, with lower efficiency.

If it is important to achieve survey uniformity after 3 years, the
available visits for each roll would be reduced also.  With a 2$\times$
enhancement of epoch frequency, a roll would last 2 months.

Some leverage would be gained by using more than 10\% of the available
visits for a single roll.  However, this begins to impact the sampling
of slow variables reduce schedule flexibility and robustness, and should
be approached with caution.

From these examples, it appears that a 2$\times$ enhancement with
uniformity closure after 10 years is relatively feasible and promising.
Much higher gains, or more rapid closure, require additional
compromises.

% --------------------------------------------------------------------

\subsection{Fast Transients and Rolling Cadence}
\label{sec:rolling:transients}

Fast transients as a science topic are addressed in \autoref{chp:transients}. In this
section, the demands of fast transients are used to directly constrain
or orient the rolling cadence development.

By ``fast transients'', we are referring to events that are sufficiently
fast that they are not addressed by the rolling cadence designed for SN
observations, and slow enough that they are not covered in the cosmological ``deep
drilling field'' cadences (\autoref{sec:intro:baseline}, \autoref{sec:cadexp:opsim}).  For higher tempo rolls, it is quite
difficult to obtain full color data, because of the constraints on
filter selection.  For this example, we will examine a rolling cadence
utilizing only the {\it r} and {\it i} filters, as they are used for
most visits. They are close in wavelength, and we assume that sufficient
color information will be obtained by the ``background'' uniform survey
that continues during a roll.

Again using 10\% of the available visits from the full 10 year survey
for a single roll, we find that there would be enough epochs for each
roll to acquire 1 visit per day for 21 consecutive days, giving an
enhancement of 10$\times$.

Alternatively, the same epochs could be used to observe a target every
20 minutes for 12 hours during a single night (here it is assumed that
visit pairs are not required, doubling the available epochs) for an
enhancement of 300$\times$.

Several different possible redeployments of portions of a uniform survey
have been described, each using 10\% of available time.  Of course it is
possible in principle to implement multiple options, sequentially or
maybe in parallel in some cases. This may pose considerable challenges
to the scheduling strategy design by introducing incompatible boundary
conditions.

While rolling cadences are powerful, they have limitations.  For
example, sampling events that last longer than $\simeq$1 day and less
than $\simeq$ 1 week have the obvious problem of diurnal availability.
In this example, intermediate cadences could be implemented in the
circumpolar region, where diurnal access is much extended.  This is an
example of a case in which a mini-survey of a limited number of regions
could be considered as an alternative to a rolling cadence applied to
the entire main survey.


% --------------------------------------------------------------------

\subsection{Constraints, Trades and Compromises for Rolling Cadences}
\label{sec:rolling:trades}

While rolling cadences offer some attractive benefits, it is important
to realize that rolling cadences are very highly constrained, and that
they do bring disadvantages and compromises.

There are strong arguments against beginning a rolling cadence in the
first, or even the second year of the survey.  Early in the survey, it
is important to obtain for each field/filter combination, an adequate
number of good quality photometric images, and at least one image in
excellent seeing, to support closure of photometry reductions, generation of template images for differencing analysis, and to establish the baseline for proper motion studies.

Since many major science goals require a significant degree of survey
homogeneity, it may be advisable to implement a strategy that brings the
survey to nominal uniform depth at several times, e.g.\ after 3 or 5
years.  This would strongly constrain rolling cadences. We note that such a strategy would also allow a lot of science to be done without waiting the whole 10-years.

Some science objectives favor certain distributions of visits.  For
astrometry, visits early and late in the survey and at large parallax
factors, are beneficial.  Slow variables may benefit from uniform
spacing.  Rolling cadences might impact these constraints either
favorably or unfavorably.

Many objectives are served by randomization of observing conditions for
each field.  Some rolling cadences could tend to reduce this
randomization, for example by acquiring a large number of observations
during a meteorologically favorable or unfavorable season, or during a
period of instrument performance variance.

Dithering may prove challenging with a rolling cadence, since it reduces
temporal coverage at the boundaries of the selected sky region.  This is
negligible for small dithers, but important for large dithers, which are
under consideration; making the contiguous roll area as large as possible should mitigate this issue.

These cautions illustrate that evaluation of rolling cadences must be
based on the {\it full range of schedule performance metrics,} and not just
those targeted by rolling cadence development.


% --------------------------------------------------------------------

\subsection{Directions}
\label{sec:rolling:directions}

While preliminary experiments with rolling cadences have been carried
out with \OpSim, these experiments have significant deficiencies, and
are not suited for in-depth study as of this writing. Designing and
simulating a family of rolling cadences is one of the main goals for the
Project's ``SOCS and Scheduler'' team for \OpSim version 4.
% However, analysis of the cadences described above may guide
% development of objectives for enhancement by rolling cadence.

Rolling cadences will need to satisfy the basic survey science
requirements (including those on sky area, depth and visit count ), and
then be evaluated using the same set of metrics as for other cadences.
Of particular interest will be metrics that clearly distinguish the
gains available with rolling cadences; that is, metrics that measure
schedule performance for variable targets, and especially those with
strong sampling requirements, or more rapid variability.

% \subsubsection{Future Rolling Cadence Metrics}
%
% The following metrics, based on similar metrics developed for particular
% science objectives, may be useful in tuning rolling cadence performance:
%
% \begin{itemize}
%
% \item Observation Pairs histograms.  Visit pairs are simple and easy to
% quantify.  A pair of visits in the same filter describe a brightness
% change and constrain the rate of change. A visit pair in different
% filters (probably not coeval) constrain a color.  These metrics will
% describe how rapidly LSST can detect a change in the source fluxes. They
% will be useful for such science as early discovery of SNe, and
% identification of interesting galactic microlensing events.
%
% \item Observation Triplets histograms. Similar to Pairs, but with
% closely spaced triplets allowing detection and confirmation of transient
% events \citep[as described by][]{LundEtal2016}.
%
% \item Period determination metric.  Compute a measure of the period
% determination accuracy after a given time interval, as a function of
% period. A complete analysis would allow use of real light curves for
% various object types with realistic brightness distributions.  A simpler
% approach would use a single example light curve and a fixed measurement
% error value. A simplest approach would be based on the sampling function
% and its power spectrum \citep{LundEtal2016}.  This metric group would
% rate simulation performance for period determination of periodic
% variables - mostly stars.
%
% \item Sampling of SNe light curves.  The most complete metric would
% count the number of SNe (of various types) light curves that would be
% well sampled (according to defined criteria), based on analysis of
% realistic simulated data.  A simpler metric would be based on estimated
% sampling requirements and cadence without considering brightness. The
% latter could be used initially, pending replacement (or calibration)
% with more extensive simulations.
%
% \item Time series count. For any transient or irregular variable, the
% optimum observational data will require multiple time and color samples
% within the duration of the event. Since the category of transients may
% include discoveries, it is not possible to exhaustively specify the
% source characteristics. A time series can be defined to consist of a
% series of samples, over some total duration T, with sampling interval t
% to tolerance d. The count of time series depends on the total duration,
% the number of samples, as well as the filter selection, so the challenge
% is how to visualize this information, and how to extract useful
% characteristic summary measures of performance.  This type of data can
% be valuable for identification and study of rapid phenomena such as
% stellar flares, binary mass transfer events, and exoplanet transits, and
% for slower events such as AGN reverberation mapping.
%
% \end{itemize}

% ====================================================================

\navigationbar
