% ====================================================================
%+
% NAME:
%    magcloudsurvey.tex
%
% CHAPTER:
%    specialsurveys.tex
%
% ELEVATOR PITCH:
%
% AUTHORS:
%    David Nidever (@dnidever)
%    Knut Olsen (@knutago)
%-
% ====================================================================

\section{The Magellanic Clouds Special Survey}
\def\secname{mc}\label{sec:\secname}

\credit{dnidever},
\credit{knutago}.

An LSST survey that did not include coverage of the Magellanic Clouds
and their periphery would be tragically incomplete.  LSST has a unique
role to play in surveys of the Clouds.  First, its large $A\Omega$
will allow us to probe the thousands of square degrees that comprise
the extended periphery of the Magellanic Clouds with unprecedented
completeness and depth, allowing us to detect and map their extended
disks, stellar halos, and debris from interactions that we already
have strong evidence must exist (REFS).  Second, the ability of LSST
to map the entire main bodies in only a few pointings will allow us to
identify and classify their extensive variable source populations with
unprecedented time and areal coverage, discovering, for example,
extragalactic planets, rare variables and transients, and light echoes
from explosive events that occurred thousands of years ago (REFS).
Finally, the large number of observing opportunities that the LSST
10-year survey will provide will enable us to produce a static imaging
mosaic of the main bodies of the Clouds with extraordinary image
quality, an invaluable legacy product of LSST.

% --------------------------------------------------------------------

\subsection{A Proposed Magellanic Clouds Mini-survey}
\label{sec:\secname:proposal}

We propose two distinct mini-surveys to meet the goals of LSST
Magellanic Clouds science:
\begin{itemize}
\item A mini-survey covering the 2700$\deg^2$ with $\delta < -60$ to
the standard LSST single-exposure depth and to stacked depths of XXX,
with cadence sufficient to detect and measure light curves of RR Lyrae
stars.
\item A mini-survey covering $\sim$250$\deg^2$ of the main bodies of
the Clouds with cadence sufficient to detect exoplanet transits and
other variable objects; a subset of these images should be taken with
seeing of $0.5\arcsec$, with stacked depth reaching the confusion
limits in the Clouds.
\end{itemize}

Figure X shows a rough map of the proposed mini-surveys.
% Need the figure and caption


% --------------------------------------------------------------------

\subsection{Mini-survey Impact on the Magellanic Cloud Science Projects}
\label{sec:\secname:revisit}

\new{Here we revisit the metric analysis of the Magellanic  Clouds'
science cases (\autoref{chp:MCs}), and make some predictions about how
they are likely to improve given  the above proposal.}


% --------------------------------------------------------------------

\subsection{Discussion}
\label{sec:\secname:discussion}


% ====================================================================
%
% \subsection{Conclusions}
%
% Here we answer the ten questions posed in
% \autoref{sec:intro:evaluation:caseConclusions}:
%
% \begin{description}
%
% \item[Q1:] {\it Does the science case place any constraints on the
% tradeoff between the sky coverage and coadded depth? For example, should
% the sky coverage be maximized (to $\sim$30,000 deg$^2$, as e.g., in
% Pan-STARRS) or the number of detected galaxies (the current baseline 
% of 18,000 deg$^2$)?}
%
% \item[A1:] ...
%
% \item[Q2:] {\it Does the science case place any constraints on the
% tradeoff between uniformity of sampling and frequency of  sampling? For
% example, a rolling cadence can provide enhanced sample rates over a part
% of the survey or the entire survey for a designated time at the cost of
% reduced sample rate the rest of the time (while maintaining the nominal
% total visit counts).}
%
% \item[A2:] ...
%
% \item[Q3:] {\it Does the science case place any constraints on the
% tradeoff between the single-visit depth and the number of visits
% (especially in the $u$-band where longer exposures would minimize the
% impact of the readout noise)?}
%
% \item[A3:] ...
%
% \item[Q4:] {\it Does the science case place any constraints on the
% Galactic plane coverage (spatial coverage, temporal sampling, visits per
% band)?}
%
% \item[A4:] ...
%
% \item[Q5:] {\it Does the science case place any constraints on the
% fraction of observing time allocated to each band?}
%
% \item[A5:] ...
%
% \item[Q6:] {\it Does the science case place any constraints on the
% cadence for deep drilling fields?}
%
% \item[A6:] ...
%
% \item[Q7:] {\it Assuming two visits per night, would the science case
% benefit if they are obtained in the same band or not?}
%
% \item[A7:] ...
%
% \item[Q8:] {\it Will the case science benefit from a special cadence
% prescription during commissioning or early in the survey, such as:
% acquiring a full 10-year count of visits for a small area (either in all
% the bands or in a  selected set); a greatly enhanced cadence for a small
% area?}
%
% \item[A8:] ...
%
% \item[Q9:] {\it Does the science case place any constraints on the
% sampling of observing conditions (e.g., seeing, dark sky, airmass),
% possibly as a function of band, etc.?}
%
% \item[A9:] ...
%
% \item[Q10:] {\it Does the case have science drivers that would require
% real-time exposure time optimization to obtain nearly constant
% single-visit limiting depth?}
%
% \item[A10:] ...
%
% \end{description}
% ====================================================================

\navigationbar
