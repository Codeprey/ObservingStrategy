% ====================================================================
%+
% NAME:
%    M67_special.tex
%
% CHAPTER:
%    specialsurveys.tex
%
% ELEVATOR PITCH:
%    As coeval, equidistant, and chemically homogeneous collections of stars,
%    open star clusters are ideal for studying the dependence of astrophysical
%    phenomena on the most fundamental stellar parameters - age and mass.
%
% AUTHORS:
%    Suzanne Hawley
%    Ruth Angus
%    Derek Buzasi
%    Jim Davenport
%    Mark Giampapa
%    Vinay Kashyap
%    Soren Meibom
%-
% ====================================================================

\section{A Mini-Survey of the Old Open Cluster M67}
\def\secname{M67_special}\label{sec:\secname}

\credit{suzannehawley},
\credit{ruthangus},
\credit{derekbuzasi},
\credit{jimdavenport},
\credit{markgiampapa},
\credit{vinaykashyap},
\credit{sorenmeibom}.

% LSST Review by Jason Kalirai:
% The kinds of effects that are being measured are things that will vary with age and mass, so you really want to survey a range of clusters with different ages. There are other clusters that satisfy the selection criteria (low extinction, well populated, nearby) that could also be added to the science case. I think building this up to a more comprehensive program offers some advantages. Otherwise, it seems to be the type of thing that could be done today with Subaru/HSC as a stand alone proposal.
% PJM: added footnote in the Science Case section.

\subsection{Introduction}

As coeval, equidistant, and chemically homogeneous collections of stars, open
star clusters are ideal for studying the dependence of astrophysical phenomena
on the most fundamental stellar parameters - age and mass.
Indeed, there are few fields in astronomy that do not rely on results from
cluster studies, and clusters play a central role in establishing how stellar
rotation and magnetic activity can be used to constrain the ages of stars and
stellar populations.
From an observational perspective, because of their angular extent they are
accessible to efficient surveys in both imaging and multi-object spectroscopy.
A selection of clusters representing a sequence in age can be used to
establish critical empirical relationships such as the dependence of activity
on rotation, the relationships between activity, rotation and stellar age, the
evolution of activity cycles, and the nature and evolution of flare
activity\textemdash{}an urgent area of investigation in view of the potential
impacts on the structure and evolution of exoplanet atmospheres in systems
with late-type host stars.

Unfortunately for observers, open clusters dissipate on timescales which are
generally comparable to stellar evolution timescales on the lower main
sequence, so older clusters are relatively rare.
In addition, most clusters lie close to the galactic plane, where determining
membership is significantly complicated by the large numbers of foreground and
background stars.

In this section, we suggest an LSST survey of M67, an open cluster whose
relative compactness, age, and location above the galactic plane combine to
make it the ideal cluster for a closer look. This proposed program could be adapted to include more clusters; we leave that more ambitious investigation to future work.

\subsection{Science Case }

The evolution of the rotation rate and magnetic activity in solar-type
stars are intimately connected. Stellar rotation drives a magnetic
dynamo, producing a surface magnetic field and magnetic activity which
manifests as starspots, chromospheric (Ca II HK, H$\alpha$) and coronal
(X-ray) emission, and flares. The magnetic field also drives a stellar
wind causing angular momentum loss (\textquotedblleft{}magnetic braking'')
which in turn slows the rotation rate over time, leading to decreased
magnetic activity. More magnetically active stars (larger spots, stronger
Ca II HK, H$\alpha$ and X-ray emission, more flares) therefore tend
to be younger and to rotate faster. The rotation-age relationship
is known as gyrochronology, and the correlation between rotation,
age and magnetic activity for solar-type stars was first codified
by Skumanich (1972). However, the decrease in rotation rate and magnetic
field strength over long time-scales is poorly understood and, in
some cases, hotly contested (Angus et al. 2015, Van Saders et al.
2016). Recent asteroseismic data from the Kepler spacecraft have revealed
that magnetic braking may cease at around solar Rossby number, implying
that gyrochronology is not applicable to older stars (Van Saders 2016).

In addition, the rotational behavior of lower mass stars is largely
unknown due to the faintness of mid-late type M dwarfs. There is reason
to believe that M dwarfs cooler than spectral type $~\mathrm{M}4$
may behave differently from the G, K and early M stars, since that
spectral type marks the boundary where the star becomes fully convective,
and a solar-type shell dynamo (which requires an interface region
between the convective envelope and radiative core of the star) can
no longer operate. Using chromospheric H$\alpha$ emission as a proxy,
West et al. (2008) studied a large sample of M dwarfs from SDSS and
showed that magnetic activity in mid-late M dwarfs lasts much longer
than in the earlier type stars.

The difficulties inherent in understanding the evolution of stellar
rotation and activity on the lower main sequence are further increased
by our inability to obtain accurate ages for field stars with ages
comparable to that of the Sun, which appears to be just the range
of ages for which our understanding of the phenomena are most suspect.
While asteroseismology can address this situation with exquisite precision,
it can only do so for the brighter stars accessible to space missions
such as Kepler. Making use of older open clusters is a way to fill
this gap.

The solar-age and solar-metallicity open cluster, M67, is a benchmark
cluster for understanding stellar evolution and the nature of late-type
stars at solar age. M67 is unique due to its solar chemical composition,
the fact that it is relatively nearby ($\sim900$ pc), and its relatively
low extinction due to its location above the galactic plane. Extensive
proper motion, radial velocity and photometric surveys have been carried
out (e.g., Girard et al. 1989, Montgomery et al. 1993, Yadav et al.
2008, Geller et al. 2015), while Giampapa et al. (2006) conducted
a survey of chromospheric activity in the solar-type members of M67
which yielded interesting insights on the range of magnetic activity
on sun-like stars in comparison with the range exhibited by the Sun
during the sunspot cycle. Nehag et al. (2011) find that solar twins
in M67 have photospheric spectra that are virtually indistinguishable
from the Sun\textquoteright{}s at echelle resolutions.

Located in the sky at approximately $\mathrm{RA}=9\mathrm{h}$ and
$\mathrm{Dec}=+12^{\circ}$, M67 is an exceptionally meritorious and
accessible candidate for an LSST special survey, which would also enable
productive follow-up observations by an array of OIR facilities.
% PJM: Note following Jason Kalirai's LSST Review:
(While we would like to study multiple clusters with LSST, we focus here on M67 in the special survey section because it is very slightly outside of the nominal footprint of the survey. Extending this science case to multiple clusters is a topic for future work.) LSST
observations of M67 would yield data on the rotation periods and variability
of its members at high precisions, particularly for dwarfs later than
about K0 ($V>16$). Little is known about the nature of variability
on short and long time scales for low-mass dwarfs at solar age. For
example, the frequency of \textquoteleft{}superflaring\textquoteright{}
at solar age could be investigated for the first time. Furthermore,
the combination of LSST observations and OIR synoptic datasets for
M67 would enable the characterization of the conditions of the habitable
zones in late- type stars at solar age.

In addition to sun-like stars, M67 includes an array of interesting
objects such as blue stragglers (Shetrone \& Sandquist 2000), an AM
Her star (Gilliland et al. 1991, Pasquini et al. 1994), a red straggler,
two subgiants (Mathieu et al. 2003), and detected X-ray sources due
to stellar coronal emission (e.g., Pasquini \& Belloni 1998). Davenport
\& Sandquist (2010) found a minimum binary fraction of 45\% in the
cluster. Other investigations include studies of the white dwarf cooling
sequence (Richer et al. 1998), angular momentum evolution near the
turnoff (Melo et al. 2001), and the behavior of key light elements
such as lithium and beryllium (e.g., Randich et al. 2007).


\subsection{Observing Plans }

Performing the special survey of M67 which we advocate would require
two modifications to the baseline LSST operations mode. LSST does
not plan to observe as far north as $\mathrm{Dec}=+12^{\circ}$ in
its main survey, but the M67 field should certainly be accessible
for a special survey as a single pointing. Since imaging the entire cluster
would require less than a single LSST field, we view this additional
pointing as being of minimal inconvenience relative to the expected
scientific gain. As we anticipate rotation periods ranging from $\sim\mathrm{days}$
up to several months, we would require sampling over all of these
timescales, though it need not be continuous.

A second potential complication is that the cluster is relatively
bright. While dwarfs below about spectral K0 in M67 are fainter than
the LSST bright limit of $\sim16$, the cluster G dwarfs will saturate
the LSST detectors in a 15-second integration. We suggest two alternative
approaches to address this issue. First, if the short exposure surveying mode
suggested in \autoref{sec:shortexp} is adopted,
then the new LSST minimum exposure time of 0.1 seconds would easily
accommodate the entire M67 main sequence. Alternatively, or if the
short exposure mode is not adopted, we note that work with the Kepler
mission (e.g., Haas et al. 2011) has shown success using custom pixel
masks to accurately perform photometry on stars as much as 6 magnitudes
brighter than the saturation level. Similar techniques applied to
the LSST fields should enable photometry for the G dwarfs, particularly
those in less-crowded portions of the field.

% ====================================================================
%
% \subsection{Conclusions}
%
% Here we answer the ten questions posed in
% \autoref{sec:intro:evaluation:caseConclusions}:
%
% \begin{description}
%
% \item[Q1:] {\it Does the science case place any constraints on the
% tradeoff between the sky coverage and coadded depth? For example, should
% the sky coverage be maximized (to $\sim$30,000 deg$^2$, as e.g., in
% Pan-STARRS) or the number of detected galaxies (the current baseline
% of 18,000 deg$^2$)?}
%
% \item[A1:] ...
%
% \item[Q2:] {\it Does the science case place any constraints on the
% tradeoff between uniformity of sampling and frequency of  sampling? For
% example, a rolling cadence can provide enhanced sample rates over a part
% of the survey or the entire survey for a designated time at the cost of
% reduced sample rate the rest of the time (while maintaining the nominal
% total visit counts).}
%
% \item[A2:] ...
%
% \item[Q3:] {\it Does the science case place any constraints on the
% tradeoff between the single-visit depth and the number of visits
% (especially in the $u$-band where longer exposures would minimize the
% impact of the readout noise)?}
%
% \item[A3:] ...
%
% \item[Q4:] {\it Does the science case place any constraints on the
% Galactic plane coverage (spatial coverage, temporal sampling, visits per
% band)?}
%
% \item[A4:] ...
%
% \item[Q5:] {\it Does the science case place any constraints on the
% fraction of observing time allocated to each band?}
%
% \item[A5:] ...
%
% \item[Q6:] {\it Does the science case place any constraints on the
% cadence for deep drilling fields?}
%
% \item[A6:] ...
%
% \item[Q7:] {\it Assuming two visits per night, would the science case
% benefit if they are obtained in the same band or not?}
%
% \item[A7:] ...
%
% \item[Q8:] {\it Will the case science benefit from a special cadence
% prescription during commissioning or early in the survey, such as:
% acquiring a full 10-year count of visits for a small area (either in all
% the bands or in a  selected set); a greatly enhanced cadence for a small
% area?}
%
% \item[A8:] ...
%
% \item[Q9:] {\it Does the science case place any constraints on the
% sampling of observing conditions (e.g., seeing, dark sky, airmass),
% possibly as a function of band, etc.?}
%
% \item[A9:] ...
%
% \item[Q10:] {\it Does the case have science drivers that would require
% real-time exposure time optimization to obtain nearly constant
% single-visit limiting depth?}
%
% \item[A10:] ...
%
% \end{description}

\navigationbar
