\setcounter{chapter}{0}
\chapter*{Summary}
\def\chpname{summary}\label{chp:\chpname}
\addcontentsline{toc}{section}{Summary}
\markboth{}{}

% Short (c. 250 word) summary for arxiv posting:

\noindent
The Large Synoptic Survey Telescope is designed to provide an 
unprecedented optical imaging dataset that will support investigations 
of our Solar System, Galaxy and Universe, across half the sky and over ten
years of repeated observation.
%
However, exactly how the LSST observations will be taken (the observing strategy or ``cadence'') is not yet finalized.
%
In this dynamically-evolving community white paper, we explore how the
detailed performance of the anticipated science investigations is
expected to depend on small changes to the LSST observing strategy.
%
Using realistic simulations of the LSST schedule and observation
properties, we design and compute diagnostic metrics and Figures of
Merit that provide quantitative evaluations of different observing
strategies, analyzing their impact on a wide range of proposed science
projects.
%
This is work in progress: we are using this white paper to communicate
to each other the relative merits of the observing strategy choices that
could be made, in an effort to maximize the scientific value of the
survey.
%
The investigation of some science cases leads to suggestions for new
strategies that could be simulated and potentially adopted.
%
Notably, we find motivation for exploring departures from a spatially
uniform annual tiling of the sky: focusing instead on different parts of
the survey area in different years in a ``rolling cadence'' is likely to
have significant benefits for a number of time domain and moving object
astronomy projects.
%
The communal assembly of a suite of quantified and homogeneously coded
metrics is the vital first step towards an automated, systematic,
science-based assessment of any given cadence simulation, that will
enable the scheduling of the LSST to be as well-informed as possible.

\clearpage
