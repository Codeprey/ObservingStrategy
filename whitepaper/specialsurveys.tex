\chapter[Special Surveys]{Special Surveys}
\def\chpname{specialsurveys}\label{chp:\chpname}

Chapter editors:
\credit{dnidever},
\credit{knutago}.

% Confirmed leads for LMC/SMC: Knut Olsen, David Nidever

% Confirmed leads for special surveys:

% \section*{Summary}
% \addcontentsline{toc}{section}{~~~~~~~~~Summary}
%
% Executive summary goes here, highlighting the primary conclusions from
% the chapter's science cases. This should be abstract length, no more:
% say, 200 words.

% --------------------------------------------------------------------

\section{Introduction}
\label{sec:specials:intro}

The four main LSST science themes, as defined by the Science Book,
drive the design of LSST's main Wide-Fast-Deep survey.  However, it
has always been recognized that many important scientific projects,
including some that are highly relevant to LSST's main science themes,
are not well served by the areal coverage and/or cadence constraints
placed on the WFD survey.  To this end, the LSST Project set aside a
nominal 10\% of the observing time to serve what are collectively
called ``special surveys.''

Projects that
will certainly make use of this $\approx10\%$ time (that is not dedicated to the WFD
survey) include the Deep Drilling fields and the Galactic Plane surveys,
as well as any survey wishing to
observe at declinations below $-60^\circ$, such as the Magellanic
Clouds.  These special programs have the potential to
heavily oversubscribe the nominal 10\%
of time assigned to them.  It is of thus critical importance for these
programs to define compelling science cases, clearly justify their
observing requirements, and derive metrics to quantify the performance
of a given schedule for the program. This chapter provides a venue for
such investigations.

A minimal set of 4 ``extragalactic'' Deep Drilling Fields have been
included in many of the \OpSim runs to date, including the baseline cadence simulation, \opsimdbref{db:baseCadence} (\autoref{chp:cadexp}), and have
been evaluated in various science sections throughout this paper.
As described in \autoref{sec:intro:timeline}, there will be a further call for Deep Drilling Field proposals, and so we defer discussion of such observing programs to that activity. In this chapter we collect together various ideas for additional special surveys or otherwise unusual observing programs, and the discussion of their metric evaluation.

In \autoref{sec:solar_system_specials} a number of different concepts for special surveys that would support solar system science is described. This is followed by an extensive discussion of short exposure observing at twighlight (\autoref{sec:shortexp}). If short exposure times will indeed be a possibility, it won't just be solar system science that benefits: an example of a small program that could make good use of such a capability is the open star cluster special survey proposed in \autoref{sec:M67_special}. We expect the ideas in this chapter to be developed (or discarded) over the time before the survey starts: this white paper aims to provide a (potentially temporary) home for them during that period.


% Add sections below, one science investigation per section, one
% section per file.

% --------------------------------------------------------------------

% PJM: commented out for now, for lack of content:
% \input{SpecialSurveys/deepdrilling.tex}

% --------------------------------------------------------------------

% PJM: This is currently in the Magellanic Clouds Chapter, but
% could be moved back here soon...
% % ====================================================================
%+
% NAME:
%    magcloudsurvey.tex
%
% CHAPTER:
%    specialsurveys.tex
%
% ELEVATOR PITCH:
%
% AUTHORS:
%    David Nidever (@dnidever)
%    Knut Olsen (@knutago)
%-
% ====================================================================

\section{The Magellanic Clouds Special Survey}
\def\secname{mc}\label{sec:\secname}

\credit{dnidever},
\credit{knutago}.

An LSST survey that did not include coverage of the Magellanic Clouds
and their periphery would be tragically incomplete.  LSST has a unique
role to play in surveys of the Clouds.  First, its large $A\Omega$
will allow us to probe the thousands of square degrees that comprise
the extended periphery of the Magellanic Clouds with unprecedented
completeness and depth, allowing us to detect and map their extended
disks, stellar halos, and debris from interactions that we already
have strong evidence must exist (REFS).  Second, the ability of LSST
to map the entire main bodies in only a few pointings will allow us to
identify and classify their extensive variable source populations with
unprecedented time and areal coverage, discovering, for example,
extragalactic planets, rare variables and transients, and light echoes
from explosive events that occurred thousands of years ago (REFS).
Finally, the large number of observing opportunities that the LSST
10-year survey will provide will enable us to produce a static imaging
mosaic of the main bodies of the Clouds with extraordinary image
quality, an invaluable legacy product of LSST.

% --------------------------------------------------------------------

\subsection{A Proposed Magellanic Clouds Special Survey}
\label{sec:\secname:proposal}

We propose two distinct special surveys to meet the goals of LSST
Magellanic Clouds science:
\begin{itemize}
\item A special survey covering the 2700$\deg^2$ with $\delta < -60$ to
the standard LSST single-exposure depth and to stacked depths of XXX,
with cadence sufficient to detect and measure light curves of RR Lyrae
stars.
\item A special survey covering $\sim$250$\deg^2$ of the main bodies of
the Clouds with cadence sufficient to detect exoplanet transits and
other variable objects; a subset of these images should be taken with
seeing of $0.5\arcsec$, with stacked depth reaching the confusion
limits in the Clouds.
\end{itemize}

Figure X shows a rough map of the proposed special surveys.
% Need the figure and caption


% --------------------------------------------------------------------

\subsection{Special Survey Impact on the Magellanic Cloud Science Projects}
\label{sec:\secname:revisit}

\new{Here we revisit the metric analysis of the Magellanic  Clouds'
science cases (\autoref{chp:MCs}), and make some predictions about how
they are likely to improve given  the above proposal.}


% --------------------------------------------------------------------

\subsection{Discussion}
\label{sec:\secname:discussion}


% ====================================================================
%
% \subsection{Conclusions}
%
% Here we answer the ten questions posed in
% \autoref{sec:intro:evaluation:caseConclusions}:
%
% \begin{description}
%
% \item[Q1:] {\it Does the science case place any constraints on the
% tradeoff between the sky coverage and coadded depth? For example, should
% the sky coverage be maximized (to $\sim$30,000 deg$^2$, as e.g., in
% Pan-STARRS) or the number of detected galaxies (the current baseline
% of 18,000 deg$^2$)?}
%
% \item[A1:] ...
%
% \item[Q2:] {\it Does the science case place any constraints on the
% tradeoff between uniformity of sampling and frequency of  sampling? For
% example, a rolling cadence can provide enhanced sample rates over a part
% of the survey or the entire survey for a designated time at the cost of
% reduced sample rate the rest of the time (while maintaining the nominal
% total visit counts).}
%
% \item[A2:] ...
%
% \item[Q3:] {\it Does the science case place any constraints on the
% tradeoff between the single-visit depth and the number of visits
% (especially in the $u$-band where longer exposures would minimize the
% impact of the readout noise)?}
%
% \item[A3:] ...
%
% \item[Q4:] {\it Does the science case place any constraints on the
% Galactic plane coverage (spatial coverage, temporal sampling, visits per
% band)?}
%
% \item[A4:] ...
%
% \item[Q5:] {\it Does the science case place any constraints on the
% fraction of observing time allocated to each band?}
%
% \item[A5:] ...
%
% \item[Q6:] {\it Does the science case place any constraints on the
% cadence for deep drilling fields?}
%
% \item[A6:] ...
%
% \item[Q7:] {\it Assuming two visits per night, would the science case
% benefit if they are obtained in the same band or not?}
%
% \item[A7:] ...
%
% \item[Q8:] {\it Will the case science benefit from a special cadence
% prescription during commissioning or early in the survey, such as:
% acquiring a full 10-year count of visits for a small area (either in all
% the bands or in a  selected set); a greatly enhanced cadence for a small
% area?}
%
% \item[A8:] ...
%
% \item[Q9:] {\it Does the science case place any constraints on the
% sampling of observing conditions (e.g., seeing, dark sky, airmass),
% possibly as a function of band, etc.?}
%
% \item[A9:] ...
%
% \item[Q10:] {\it Does the case have science drivers that would require
% real-time exposure time optimization to obtain nearly constant
% single-visit limiting depth?}
%
% \item[A10:] ...
%
% \end{description}
% ====================================================================

\navigationbar


% --------------------------------------------------------------------

% ====================================================================
%+
% NAME:
%    section-name.tex
%
% ELEVATOR PITCH:
%    Explain in a few sentences what the relevant discovery or
%    measurement is going to be discussed, and what will be important
%    about it. This is for the browsing reader to get a quick feel
%    for what this section is about.
%
% COMMENTS:
%
%
% BUGS:
%
%
% AUTHORS:
%    David Nidever (@dnidever)
%    Knut Olsen (@knutago)
%-
% ====================================================================

\section{Solar System Special Surveys}
\def\secname{solar_system_specials}\label{sec:\secname}

\credit{davidtrilling},
\credit{rhiannonlynne}.

% This individual section will need to describe the particular
% discoveries and measurements that are being targeted in this section's
% science case. It will be helpful to think of a ``science case" as a
% ``science project" that the authors {\it actually plan to do}. Then,
% the sections can follow the tried and tested format of an observing
% proposal: a brief description of the investigation, with references,
% followed by a technical feasibility piece. This latter part will need
% to be quantified using the MAF framework, via a set of metrics that
% need to be computed for any given observing strategy to quantify its
% impact on the described science case. Ideally, these metrics would be
% combined in a well-motivated figure of merit. The section can conclude
% with a discussion of any risks that have been identified, and how
% these could be mitigated.

%A short preamble goes here. What's the context for this science
%project? Where does it fit in the big picture?

There are several populations of Near Earth Objects (Solar System bodies
whose orbits bring them close to the Earth's orbit) that, because of
their orbital properties, would not be easily detected in the
wide-fast-deep survey. These populations are very interesting for both
scientific and sociological purposes, though, due to their close
proximity to the Earth, and in fact their potential for impacting the
Earth. LSST will have the capability to carry out surveys for these
populations by using a small amount of special survey time. Two of
these special surveys have pointings that fall within the nominal
wide-fast-deep plan, and simply require a modification of the cadence.
The third program is a twilight program, with a special cadence (though
all twilight programs are likely to  have special cadences). These three
programs are listed here and described below. The three special surveys are
the following:

\begin{itemize}
\item A special survey to look for mini-moons, which are temporarily captured
satellites of the Earth;
\item A special survey to find meter-sized impactors up to two weeks prior to impact.
This would allow telescopic characterization of these impactors, which could
be compared to laboratory measurements of the meteorites derived from
the impactor. Advanced warning of an impactor also allows detailed
study of impact physics by being on location when the impact
occurs;
\item A special survey to observe the ``sweetspot'' in twilight fields
to look for NEOs in very Earth-like orbits that would otherwise not
be found in opposition fields.
\end{itemize}

% Need the figure and caption
These surveys will support three important scientific investigations:
\begin{enumerate}
\item What are the properties of the population of objects that is
nearest to the Earth?
\item What is the impact risk from NEOs in populations that
have not yet been well characterized (mini-moons, sweetspot objects)?
\item How do the telescopic properties of an impactor relate to the
laboratory-measured properties of the ensuing meteorites?
\end{enumerate}

%Many different types of objects and measurements with their own cadence
%``requirements'' will fall into these two broad categories (with some
%overlap).  These will be outlined in the next section.
Some details of the special cadence requirements for these
science investigations are described in the following section.

% --------------------------------------------------------------------

\subsection{Target measurements and discoveries}
\label{sec:\secname:targets}

\subsubsection{Special cadences}

Each of the three Solar System special surveys requires a special
cadence. These cadences are described here.

\begin{itemize}

\item{{\bf Mini-moons}}
Mini-moons are objects that are temporary satellites of the Earth
\citep{2014Icar..241..280B, 2017Icar..285...83F}
% bolin et al, icarus, 241, 280 http://adsabs.harvard.edu/abs/2014Icar..241..280B
% fedorets et al. icarus 285 83 http://adsabs.harvard.edu/abs/2017Icar..285...83F
Therefore, they have orbital motions similar to the Earth's moon,
and much faster than other Solar System populations. Therefore,
a special cadence is required to detect these objects enough
times to link objects, create tracklets, and determine orbits.
A suggested cadence for a mini-moon survey is a series
of 3~second exposures, with each pointing visited at least
twice per night. Such a survey would cover essentially
all of the opposition sky each night. The opposition sky should
be re-observed several nights in a row in order to
link objects from night to night and determine their orbits.
While the details of this special cadence are not yet
fully refined, this special survey would likely have little impact
on the overall LSST program since this small-scale
program, which extends over a small number of nights,
is effectively a compressed rolling cadence in which
the aggregate field coverage is unchanged.

\item{{\bf Impactors}}
The Earth is struck by meter-sized impactors about
once a month \citep[\eg][Trilling \etal 2017 submitted]{Boslough2015, 2017Icar..284..416T}.
% boslough et al. 2015 in Aerospace Conference, 2015 IEEE, 1-12
% tricarico 2017 icarus 284 416
% trilling et al 2017 AAS journals submitted
On two occasions, impacting asteroids have
been discovered some hours before impact, but
there are no existing surveys that are dedicated to finding
impactors.
% xxx ATLAS xxx.
Impactors generally have small apparent motions
on the sky (because their orbits are not too different
than the Earth's). The single exposure depth of LSST
images suggests that a meter-sized NEO could be
discovered perhaps a week before impact, given
the typical Earth-relative velocity of such a body
\citep[\eg][]{2017arXiv170506209C}.
% chesley & veres 2017 https://arxiv.org/abs/1705.06209
A suggested cadence for an impactor survey would be
to survey the opposition patch four times per night.
This is more visits than in the nominal cadence, and
would allow high fidelity linking of observations to
find orbits. The nominal wide-fast-deep cadence
(twice per night, three times during a lunation) has
a latency of orbit determination of up to two weeks,
which is not acceptable for the impactor survey, as an
impact would occur on a timescale of just a few days
from discovery.
The cadence of four observations/night should be repeated
roughly every three days, so that an object on an
inbound trajectory could be observed at least once,
and possibly twice, before impact.
Note that this cadence is compatible with
the wide-fast-deep survey, in that the fields and
exposure times are nominal; the only difference is that
each field is visited four times in a night, and that
the fields are revisited every few nights. The overall
impact of this special survey on the wide-fast-deep
survey is likely to be small, and possibly negligible.
Given the importance of this small but
significant investigation, it is critical that the
survey simulators be capable of including such
a special survey in planning for LSST operations.

\item{{\bf Twilight/sweetspot survey}}

NEOs on very Earth-like orbits are relatively
unlikely to come to opposition, and therefore
are relatively unlikely to appear in data
obtained in the wide-fast-deep survey.
These objects are particularly interesting
since, having very Earth-like orbits, they
are the most likely objects to be Earth
impactors.
These objects are most likely to be detected
in a twilight survey that looks at the ``sweetspot'' ---
a location at around 60~degrees Solar
elongation that is only visible at twilight.
Because these sweetspot fields are only visible
for 30--60~twilight minutes each night,
a special
cadence is required to find and link these objects
to determine their orbits.
These observations would be best carried out
in the $z$ filter (because the observations are
made in twilight, when the sky is still relatively
bright). Fields should be revisited at 15~minute
intervals, and each field should be revisited
every other night during this experiment, so that
observations can be linked.
(A long interval
between observations prohibits linking.)
The total experiment
should last roughly one week, so that each
object would have a tracklet on four nights
(nights 1,3,5,7).
During twilight, some 25~pointings could be visited
after the sky is sufficiently dark but
before the fields have set.
Because these observations are made during twilight,
there may be no significant impact on the
nominal wide-fast-deep survey. See \autoref{sec:shortexp} for a more extensive discussion of short exposure and twilight observing in special survey mode.
\end{itemize}

\subsubsection{Measurements}

For each of these three programs, the most important measurement
to be made is the position of any object as a function of time.
In other words, the usual measurements of moving
objects from LSST images is also the requirement for
the source detections for these special surveys. As usual
for Solar System surveys, there is a trade-off of
sensitivity (Solar System objects are most easily
detected in $r$ band) against characterization (observing
a given object in multiple filters yields an estimate
of composition). For these three cases, discovery and
good orbit determination is probably more important than
immediate characterization from LSST measurements,
so the nominal expectation is that
the nighttime special surveys would be carried out in
$r$ band and the twilight program in $z$ band.


% --------------------------------------------------------------------

\subsection{Metrics}
\label{sec:\secname:metrics}

The metrics to be used to determine the efficacy of LSST
at scientific success of these special surveys are identical
to those employed in \autoref{chp:solarsystem}.
The most important of these metrics
include the completeness as a function of size; the
number of detections over a given length of time (for instance,
the one week approach timescale of impactors); and
the quality of the derived orbit. These metrics are defined
in more detail in \autoref{chp:solarsystem}. The important question is:
how much value do the special surveys add?

%
% % --------------------------------------------------------------------
%
% \subsection{OpSim Analysis}
% \label{sec:\secname:analysis}

The current default observing strategy does not include
any of these special surveys. Therefore, the scientific yield,
at this default, is zero. Both the mini-moons and impactor
surveys are relatively small experiments, on the scale of
the LSST project, at something like 10--20~hours total
per instance of the experiment. (The impactor experiment,
for example, might be carried out one or several times a year,
both to build up statistics and to identify further potential
impactors.) Furthermore, the impactors survey cadence
is different from the nominal wide-fast-deep survey,
but could be a simple modification of the nominal wide-fast-deep survey
cadence.

The twilight/sweetspot survey is also not included in
the current baseline \OpSim strategy, and nor are any twilight observations (\autoref{sec:shortexp}).
It is critical to ensure that \OpSim can handle the kind
of dedicated cadences described above in order to assess
the global impact of these small-scale but highly
important Solar System special surveys.



%
% % --------------------------------------------------------------------
%
% \subsection{Discussion}
% \label{sec:\secname:discussion}
%
% Discussion: what risks have been identified? What suggestions could be
% made to improve this science project's figure of merit, and mitigate
% the identified risks?
%

% ====================================================================

% ====================================================================
%
% \subsection{Conclusions}
%
% Here we answer the ten questions posed in
% \autoref{sec:intro:evaluation:caseConclusions}:
%
% \begin{description}
%
% \item[Q1:] {\it Does the science case place any constraints on the
% tradeoff between the sky coverage and coadded depth? For example, should
% the sky coverage be maximized (to $\sim$30,000 deg$^2$, as e.g., in
% Pan-STARRS) or the number of detected galaxies (the current baseline
% of 18,000 deg$^2$)?}
%
% \item[A1:] ...
%
% \item[Q2:] {\it Does the science case place any constraints on the
% tradeoff between uniformity of sampling and frequency of  sampling? For
% example, a rolling cadence can provide enhanced sample rates over a part
% of the survey or the entire survey for a designated time at the cost of
% reduced sample rate the rest of the time (while maintaining the nominal
% total visit counts).}
%
% \item[A2:] ...
%
% \item[Q3:] {\it Does the science case place any constraints on the
% tradeoff between the single-visit depth and the number of visits
% (especially in the $u$-band where longer exposures would minimize the
% impact of the readout noise)?}
%
% \item[A3:] ...
%
% \item[Q4:] {\it Does the science case place any constraints on the
% Galactic plane coverage (spatial coverage, temporal sampling, visits per
% band)?}
%
% \item[A4:] ...
%
% \item[Q5:] {\it Does the science case place any constraints on the
% fraction of observing time allocated to each band?}
%
% \item[A5:] ...
%
% \item[Q6:] {\it Does the science case place any constraints on the
% cadence for deep drilling fields?}
%
% \item[A6:] ...
%
% \item[Q7:] {\it Assuming two visits per night, would the science case
% benefit if they are obtained in the same band or not?}
%
% \item[A7:] ...
%
% \item[Q8:] {\it Will the case science benefit from a special cadence
% prescription during commissioning or early in the survey, such as:
% acquiring a full 10-year count of visits for a small area (either in all
% the bands or in a  selected set); a greatly enhanced cadence for a small
% area?}
%
% \item[A8:] ...
%
% \item[Q9:] {\it Does the science case place any constraints on the
% sampling of observing conditions (e.g., seeing, dark sky, airmass),
% possibly as a function of band, etc.?}
%
% \item[A9:] ...
%
% \item[Q10:] {\it Does the case have science drivers that would require
% real-time exposure time optimization to obtain nearly constant
% single-visit limiting depth?}
%
% \item[A10:] ...
%
% \end{description}

\navigationbar


% --------------------------------------------------------------------

% ====================================================================
%+
% NAME:
%    short_exposures.tex
%
% CHAPTER:
%    specialsurveys.tex
%
% ELEVATOR PITCH:
%
% AUTHORS:
%    Chris Stubbs (@astrostubbs))
%-
% ====================================================================

\section{Short Exposure Surveying}
\def\secname{shortexp}\label{sec:\secname}

\credit{astrostubbs}

The current LSST requirements stipulate a minimum exposure time of 5
seconds, with an expected default exposure time of 15 seconds. This
document advocates for decreasing the minimum exposure time requirement
from 5 to 0.1 seconds. This would increase the dynamic range for bright
sources (compared to the default 15 sec time) by about 5 magnitudes, to
a total of 13 astronomical magnitudes (where dynamic range is the
difference between the brightest unsaturated source and the faintest
point source detectable at 5 sigma). This is a large factor, and would
enable a wide range of science goals, outlined below. One interesting
aspect of this is that it would allow us to operate the LSST system
during twilight times that would otherwise saturate the array due to
background sky brightness. This would allow a number of the goals
described below to be carried out without impacting the primary survey
by conducting observations during twilight sky conditions that would
saturate the array at longer exposure times.


% ----------------------------------------------------------------------

\subsection{Introduction}
\label{sec:\secname:intro}

Since the twilight sky brightness is an important factor discussed
below, we provide here a very brief outline of the temporal evolution of
the background sky brightness.

\citet{1993AJ....105.1206T}
provide a simple framework that serves our purposes well. They provide
observational data as well as a simple model for the evolution of
twilight sky brightness. Figure~1 from that paper is included below, as \autoref{fig:Tyson}.
They show that a good model for the sky brightness evolution is given by
an exponential with
$\log_{10}(S)=(k/\tau)t+C$,
where S is the sky brightness in electrons per pixel per second, C is
the dark sky background, k = (10.6 minutes)$^{-1}$  is a universal
(band-independent) timescale during which the sky's surface brightness
changes by a factor of ten (at latitude $-$30 degrees), and $\tau$ is a
season-dependent factor that ranges from 1.0 at the equinox to 1.07 in
austral winter and 1.20 in austral summer. So the rule of thumb is that
we should expect it to take 4.25 minutes for the sky background to
change by one magnitude per square arc sec. (In what follows we'll
ignore the increased twilight time in summer and winter.)

For current generation typical astronomical camera systems that take
over a minute to read out, this 4.2 minute time scale means that only a
handful of images can be obtained during twilight time. But for the LSST
camera with a 2 second readout time, we can obtain hundreds of short
exposures during twilight. Even if we are limited to a 15 second cadence
due to thermal stability or data transfer limitations there is a large
amount of time opened up that we can use.

What do we stand to gain in operational time with shorter exposures? If
the standard survey terminates taking 15 second exposures due to some
sky brightness criterion, by shifting to 0.1 sec images at that point we
will have changed the sky flux per pixel by 2.5 $\log_{10}(150)$ = 5.4
magnitudes. This brings us back into a high dynamic range regime, as
described below.

\begin{figure}[htbp]
\begin{center}
\includegraphics[trim = 0 7cm 0 1mm, clip, width=\textwidth]{figs/Stubbs_Fig1.pdf}
\caption{(reproduced from Tyson et al, 1993). This plot shows the
  twilight sky surface brightness as a function of local time for four
  broadband filters (C, B, V and R) and different pointing directions.
  The surface brightness changes by one magnitude in a 4.2 minute interval,
essentially independently of the passband and pointing.}
\label{fig:Tyson}
\end{center}
\end{figure}

\autoref{fig:twilight} illustrates the principles that underpin this proposal. LSST is
a unique combination of hardware and software, that will deliver
reliable catalogs of both the static and the dynamic sky. By pushing
towards shorter integration times we can greatly expand the scientific
reach of the system.

The dynamic range in magnitudes that we can achieve for a given
integration time depends on the sky background, the read noise, and the
full well depth per pixel. We will adopt a typical value of 100Ke for
the full well depth, but the arguments presented below are essentially
independent of this value. The dynamic range in magnitudes is limited on
the bright end by the point source whose PSF peak exceeds full well, and
on the faint end by the 5$\sigma$ point source sensitivity, which
depends on sky brightness per pixel. So we are squeezed between the two
parameters of full well depth and sky background.

\begin{figure}[htbp]
\begin{center}
\includegraphics[width=6in]{figs/Stubbs_Fig2.pdf}
\caption{Twilight dynamic range. As we enter morning twilight time, the increasing sky brightness requires brighter sources for 5 sigma detection, and also limits unsaturated objects to increasingly fainter sources. Eventually the gap between these goes to zero. But operating at shorter exposure times allows us to push useful survey operations into brighter twilight time, and also to increase the dynamic range of the LSST survey products. The black lines correspond to 15 second integrations (nominally in the r band), the red lines to 5 second exposures, and the blue curves to 0.1 second exposures. The upper lines in each case represent the 5 sigma point source detection threshold while the lower line corresponds to the source brightness that produces saturation in the peak pixel of the PSF. Adding shorter exposure times increases our dynamic range in flux, and adds valuable observing time.}
\label{fig:twilight}
\end{center}
\end{figure}

The 5-sigma limiting flux scales as the square root of the sky
brightness, while the saturation flux decreases linearly as sky
brightness increases. So the two curves in \autoref{fig:twilight} have
slopes that differ by a factor of two. Operating during bright-sky time
with short exposures adds about 20 minutes of observing per twilight, or
40 minutes per night. This is a non-trivial resource!

\autoref{fig:twilight} shows one reason why it is not advantageous to go
below 0.1 second exposures- we would lose the overlap between a twilight
survey and the standard LSST object catalog.


% ----------------------------------------------------------------------

\subsection{Science Drivers for Shorter Exposures}
\label{sec:\secname:drivers}

Having set the stage for the opportunity to operate at shorter exposure
times either during dark sky time, or during twilight, or both, we now
describe some of the scientific motivations for doing so.


\subsubsection{Discovery space at short time scales.}

LSST is a time domain discovery machine. It is hard to anticipate the
importance of being able to detect astronomical variability on short
time scales. By extending the time domain sensitivity to phenomena with
a characteristic time of less than 5 seconds, we will have added 1.5
orders of magnitude in time domain sensitivity.

Taking short exposures does not necessarily imply a requirement on fast
image cadence. Periodic variability can be readily detected and
characterized with a succession of short images that do not satisfy the
Nyquist criterion, as long as we know the time associated with each data
point to adequate accuracy. But it does seem appropriate to investigate
the maximum possible rapid-fire imaging rate for LSST, presumably
limited by either data transfer bottlenecks or by thermal issues within
the camera.

\subsubsection{Distances to Nearby SN Ia- an essential ingredient in using supernovae to probe dark energy.}

The determination of the equation of state parameter of the Dark Energy
using type Ia supernovae entails measuring the redshift dependence of
the luminosity distances to objects over a range of redshifts. The low
end of this redshift range is limited by peculiar velocities to
considering supernovae at redshifts z$>$0.01. At this distance (distance
modulus of $\mu$ =33) the peak brightness of a type Ia supernova is r=15
and exceeds the expected LSST point source saturation limit.

Moreover, the rate on the sky of these bright nearby supernovae is so
low that in the standard cadence we don't expect to obtain well-sampled
multiband light curves for them. But we will discover many of them on
the rise. Using twilight time with short exposures to obtain appropriate
temporal and passband coverage will allow us to extend the LSST SN
Hubble diagram across the entire redshift range of 0.01 to 1.

It is vitally important that we obtain these nearby-SN light curves on
the same photometric system, reduced with the same data reduction
pipeline, as the distant sample. This means we really must use the LSST
instrument and software in order to avoid systematic errors arising from
differences in photometric systems or algorithmic issues.

We stress that this twilight SN followup campaign can be accomplished
without impacting the main survey, during the roughly 20 minutes per
night of twilight that would otherwise unusable at the default exposure
time. We would use the brighter twilight time to obtain pointed
observations on nearby supernovae, motivated by the importance of
photometric uniformity described above.


\subsubsection{A Bright Star Survey for Galactic Science.}

We could also use the added twilight time to conduct a bright star
survey, and the precise astrometry and photometry from LSST can then be
used in conjunction with archived data ranging from 11th to 27th AB
magnitudes. This short-exposure domain would extend the LSST dynamic
range in fluxes by two orders of magnitude, towards the bright end.
Moreover, obtaining precise positions, fluxes and variability at these
brighter magnitudes would greatly increase the overlap with the
historical archive of astronomical information, including from digitized
plate data. We would be able to obtain astrometric and color information
to high precision, as well time series for variability studies.

An example of an application to Milky Way structure studies comes from
RR Lyrae variable stars. With a saturation magnitude of around 16th in
the standard LSST survey, RR Lyrae closer than 20 kpc will be saturated
in the standard LSST images. So we will lose nearly all Galactic RR
Lyrae. Extending the survey's bright limit to 11th magnitude will allow
us to collect light curves for RR Lyrae beyond $\sim$ 100 parsecs,
collecting essentially all Southern hemisphere Galactic RR Lyrae.

Another application for stellar population studies is measuring the
fraction of binary stars as a function of stellar type, metallicity, age
and environment. By conducting a variability survey in the 11-18
magnitude range we can capitalize on temperature and metallicity data
already in hand for many of these objects.

Another application of a bright star survey would be to search for
planetary transits in the magnitude range appropriate for radial
velocity followup observations using 30 meter class telescopes. For high
dispersion spectrographs at the 4m aperture class, most targets are
currently around 8th magnitude, so we should expect 30m telescopes to
attain similar radial velocity precisions for sources of magnitude  8 +
5log(30/4) = 12. By going to shorter exposures we obtain almost an
hour's additional observing time per night when these sources don't
saturate, whereas they are far beyond saturation in the default 15
second LSST survey images.

A typical (r$-$K) color between SDSS and 2MASS is r$-$K=3. The 2MASS
catalog is complete down to K$\sim$14 which corresponds to r$\sim$17. So
most 2MASS stars will be saturated in the standard LSST 15 second
observations. A bright star survey will allow a multiband match to the
2MASS data, as well as an astrometric comparison between the two
catalogs.

Finally, as pointed out in \autoref{sec:solar_system_specials}, the apparent magnitude of solar system objects depends on their
distance from us and from the sun, as well as illumination and
observation geometry. Extending the bright limit will allow us to track
asteroid positions as they approach opposition -- see \autoref{sec:solar_system_specials} for more details.


% ----------------------------------------------------------------------

\subsection{Counterarguments}
\label{sec:\secname:counter}


\subsubsection{What About Scintillation Effects?}

Short exposure times suffer from scintillation effects. An estimate for
uncertainty due to scintillation is provided by
\url{http://astro.corlan.net/gcx/scint.txt}. For a 0.1 second
integration we expect a fractional flux uncertainty of  0.15 at 2
airmasses and 0.043 at 1 airmass, for a 10 cm aperture. Scaling this up
to the 8.5m aperture of LSST by a factor D$^{2/3}$ predicts fractional
flux variations of below one percent, even at two airmasses, for a 0.1
second exposure. So scintillation should not impact our ability to make
precision measurements of flux and position.

\subsubsection{What about just doing this with smaller telescopes?}

A possible counter-argument to the proposal of allowing for shorter
exposure times is that much of this can be done with smaller telescopes.
But it's important to bear in mind that LSST is a system, and the data
reduction and dissemination tools are as important as the hardware. We
intend to deliver accessible, high-quality, well-calibrated photometry
on a common photometric system and correspondingly good positions. If we
do so from a co-added point source depth of 27th to the short-exposure
bright limit of 11th magnitude we will span over six decades in flux on
a well-calibrated flux scale. We would also have the ability to study
astrophysical variability on time scales from 0.1 second to 10 years,
which is nine decades in the time domain. This combination of temporal
and flux dynamic range would be a truly remarkable  achievement, and
would yield science benefits far beyond the illustrative examples
provided above. Much of this discovery space is enabled by going to
shorter exposures.

\subsection{Proposed Implementation and Impacts}

The implementation of this would simply entail taking short-exposure
images during twilight time that would otherwise go unused. The data
rate would go up, and the number of shutter cycles per night would also
increase, and so both the telescope and data management teams would need to advise on the cost of the program.

%
%\section{References}
%
%Tyson and Gal, An Exposure Guide for Taking Twilight Flats with Large Format CCDs, AJ {\bf 105}, 1026 (1003).

% ====================================================================
%
% \subsection{Conclusions}
%
% Here we answer the ten questions posed in
% \autoref{sec:intro:evaluation:caseConclusions}:
%
% \begin{description}
%
% \item[Q1:] {\it Does the science case place any constraints on the
% tradeoff between the sky coverage and coadded depth? For example, should
% the sky coverage be maximized (to $\sim$30,000 deg$^2$, as e.g., in
% Pan-STARRS) or the number of detected galaxies (the current baseline
% of 18,000 deg$^2$)?}
%
% \item[A1:] ...
%
% \item[Q2:] {\it Does the science case place any constraints on the
% tradeoff between uniformity of sampling and frequency of  sampling? For
% example, a rolling cadence can provide enhanced sample rates over a part
% of the survey or the entire survey for a designated time at the cost of
% reduced sample rate the rest of the time (while maintaining the nominal
% total visit counts).}
%
% \item[A2:] ...
%
% \item[Q3:] {\it Does the science case place any constraints on the
% tradeoff between the single-visit depth and the number of visits
% (especially in the $u$-band where longer exposures would minimize the
% impact of the readout noise)?}
%
% \item[A3:] ...
%
% \item[Q4:] {\it Does the science case place any constraints on the
% Galactic plane coverage (spatial coverage, temporal sampling, visits per
% band)?}
%
% \item[A4:] ...
%
% \item[Q5:] {\it Does the science case place any constraints on the
% fraction of observing time allocated to each band?}
%
% \item[A5:] ...
%
% \item[Q6:] {\it Does the science case place any constraints on the
% cadence for deep drilling fields?}
%
% \item[A6:] ...
%
% \item[Q7:] {\it Assuming two visits per night, would the science case
% benefit if they are obtained in the same band or not?}
%
% \item[A7:] ...
%
% \item[Q8:] {\it Will the case science benefit from a special cadence
% prescription during commissioning or early in the survey, such as:
% acquiring a full 10-year count of visits for a small area (either in all
% the bands or in a  selected set); a greatly enhanced cadence for a small
% area?}
%
% \item[A8:] ...
%
% \item[Q9:] {\it Does the science case place any constraints on the
% sampling of observing conditions (e.g., seeing, dark sky, airmass),
% possibly as a function of band, etc.?}
%
% \item[A9:] ...
%
% \item[Q10:] {\it Does the case have science drivers that would require
% real-time exposure time optimization to obtain nearly constant
% single-visit limiting depth?}
%
% \item[A10:] ...
%
% \end{description}

\navigationbar


% --------------------------------------------------------------------

% ====================================================================
%+
% NAME:
%    M67_special.tex
%
% CHAPTER:
%    specialsurveys.tex
%
% ELEVATOR PITCH:
%    As coeval, equidistant, and chemically homogeneous collections of stars,
%    open star clusters are ideal for studying the dependence of astrophysical
%    phenomena on the most fundamental stellar parameters - age and mass.
%
% AUTHORS:
%    Suzanne Hawley
%    Ruth Angus
%    Derek Buzasi
%    Jim Davenport
%    Mark Giampapa
%    Vinay Kashyap
%    Soren Meibom
%-
% ====================================================================

\section{A Mini-Survey of the Old Open Cluster M67}
\def\secname{M67_special}\label{sec:\secname}

\credit{suzannehawley},
\credit{ruthangus},
\credit{derekbuzasi},
\credit{jimdavenport},
\credit{markgiampapa},
\credit{vinaykashyap},
\credit{sorenmeibom}.

% LSST Review by Jason Kalirai:
% The kinds of effects that are being measured are things that will vary with age and mass, so you really want to survey a range of clusters with different ages. There are other clusters that satisfy the selection criteria (low extinction, well populated, nearby) that could also be added to the science case. I think building this up to a more comprehensive program offers some advantages. Otherwise, it seems to be the type of thing that could be done today with Subaru/HSC as a stand alone proposal.
% PJM: added footnote in the Science Case section.

\subsection{Introduction}

As coeval, equidistant, and chemically homogeneous collections of stars, open
star clusters are ideal for studying the dependence of astrophysical phenomena
on the most fundamental stellar parameters - age and mass.
Indeed, there are few fields in astronomy that do not rely on results from
cluster studies, and clusters play a central role in establishing how stellar
rotation and magnetic activity can be used to constrain the ages of stars and
stellar populations.
From an observational perspective, because of their angular extent they are
accessible to efficient surveys in both imaging and multi-object spectroscopy.
A selection of clusters representing a sequence in age can be used to
establish critical empirical relationships such as the dependence of activity
on rotation, the relationships between activity, rotation and stellar age, the
evolution of activity cycles, and the nature and evolution of flare
activity\textemdash{}an urgent area of investigation in view of the potential
impacts on the structure and evolution of exoplanet atmospheres in systems
with late-type host stars.

Unfortunately for observers, open clusters dissipate on timescales which are
generally comparable to stellar evolution timescales on the lower main
sequence, so older clusters are relatively rare.
In addition, most clusters lie close to the galactic plane, where determining
membership is significantly complicated by the large numbers of foreground and
background stars.
In this document, we suggest an LSST survey of M67, an open cluster whose
relative compactness, age, and location above the galactic plane combine to
make it the ideal cluster for a closer look.

\subsection{Science Case }

The evolution of the rotation rate and magnetic activity in solar-type
stars are intimately connected. Stellar rotation drives a magnetic
dynamo, producing a surface magnetic field and magnetic activity which
manifests as starspots, chromospheric (Ca II HK, H$\alpha$) and coronal
(X-ray) emission, and flares. The magnetic field also drives a stellar
wind causing angular momentum loss (\textquotedblleft{}magnetic braking'')
which in turn slows the rotation rate over time, leading to decreased
magnetic activity. More magnetically active stars (larger spots, stronger
Ca II HK, H$\alpha$ and X-ray emission, more flares) therefore tend
to be younger and to rotate faster. The rotation-age relationship
is known as gyrochronology, and the correlation between rotation,
age and magnetic activity for solar-type stars was first codified
by Skumanich (1972). However, the decrease in rotation rate and magnetic
field strength over long time-scales is poorly understood and, in
some cases, hotly contested (Angus et al. 2015, Van Saders et al.
2016). Recent asteroseismic data from the Kepler spacecraft have revealed
that magnetic braking may cease at around solar Rossby number, implying
that gyrochronology is not applicable to older stars (Van Saders 2016).

In addition, the rotational behavior of lower mass stars is largely
unknown due to the faintness of mid-late type M dwarfs. There is reason
to believe that M dwarfs cooler than spectral type $~\mathrm{M}4$
may behave differently from the G, K and early M stars, since that
spectral type marks the boundary where the star becomes fully convective,
and a solar-type shell dynamo (which requires an interface region
between the convective envelope and radiative core of the star) can
no longer operate. Using chromospheric H$\alpha$ emission as a proxy,
West et al. (2008) studied a large sample of M dwarfs from SDSS and
showed that magnetic activity in mid-late M dwarfs lasts much longer
than in the earlier type stars.

The difficulties inherent in understanding the evolution of stellar
rotation and activity on the lower main sequence are further increased
by our inability to obtain accurate ages for field stars with ages
comparable to that of the Sun, which appears to be just the range
of ages for which our understanding of the phenomena are most suspect.
While asteroseismology can address this situation with exquisite precision,
it can only do so for the brighter stars accessible to space missions
such as Kepler. Making use of older open clusters is a way to fill
this gap.

The solar-age and solar-metallicity open cluster, M67, is a benchmark
cluster for understanding stellar evolution and the nature of late-type
stars at solar age. M67 is unique due to its solar chemical composition,
the fact that it is relatively nearby ($\sim900$ pc), and its relatively
low extinction due to its location above the galactic plane. Extensive
proper motion, radial velocity and photometric surveys have been carried
out (e.g., Girard et al. 1989, Montgomery et al. 1993, Yadav et al.
2008, Geller et al. 2015), while Giampapa et al. (2006) conducted
a survey of chromospheric activity in the solar-type members of M67
which yielded interesting insights on the range of magnetic activity
on sun-like stars in comparison with the range exhibited by the Sun
during the sunspot cycle. Nehag et al. (2011) find that solar twins
in M67 have photospheric spectra that are virtually indistinguishable
from the Sun\textquoteright{}s at echelle resolutions.

Located in the sky at approximately $\mathrm{RA}=9\mathrm{h}$ and
$\mathrm{Dec}=+12^{\circ}$, M67 is an exceptionally meritorious and
accessible candidate for an LSST mini-survey, which would also enable
productive follow-up observations by an array of OIR facilities.
% PJM: Note following Jason Kalirai's LSST Review:
(While we would like to study multiple clusters with LSST, we focus here on M67 in the special survey section because it is very slightly outside of the nominal footprint of the survey. Extending this science case to multiple clusters is a topic for future work.) LSST
observations of M67 would yield data on the rotation periods and variability
of its members at high precisions, particularly for dwarfs later than
about K0 ($V>16$). Little is known about the nature of variability
on short and long time scales for low-mass dwarfs at solar age. For
example, the frequency of \textquoteleft{}superflaring\textquoteright{}
at solar age could be investigated for the first time. Furthermore,
the combination of LSST observations and OIR synoptic datasets for
M67 would enable the characterization of the conditions of the habitable
zones in late- type stars at solar age.

In addition to sun-like stars, M67 includes an array of interesting
objects such as blue stragglers (Shetrone \& Sandquist 2000), an AM
Her star (Gilliland et al. 1991, Pasquini et al. 1994), a red straggler,
two subgiants (Mathieu et al. 2003), and detected X-ray sources due
to stellar coronal emission (e.g., Pasquini \& Belloni 1998). Davenport
\& Sandquist (2010) found a minimum binary fraction of 45\% in the
cluster. Other investigations include studies of the white dwarf cooling
sequence (Richer et al. 1998), angular momentum evolution near the
turnoff (Melo et al. 2001), and the behavior of key light elements
such as lithium and beryllium (e.g., Randich et al. 2007).


\subsection{Observing Plans }

Performing the mini-survey of M67 which we advocate would require
two modifications to the baseline LSST operations mode. LSST does
not plan to observe as far north as $\mathrm{Dec}=+12^{\circ}$ in
its main survey, but the M67 field should certainly be accessible
for a mini-survey as a single pointing. Since imaging the entire cluster
would require less than a single LSST field, we view this additional
pointing as being of minimal inconvenience relative to the expected
scientific gain. As we anticipate rotation periods ranging from $\sim\mathrm{days}$
up to several months, we would require sampling over all of these
timescales, though it need not be continuous.

A second potential complication is that the cluster is relatively
bright. While dwarfs below about spectral K0 in M67 are fainter than
the LSST bright limit of $\sim16$, the cluster G dwarfs will saturate
the LSST detectors in a 15-second integration. We suggest two alternative
approaches to address this issue. First, if the short exposure surveying mode
suggested elsewhere in this document (Section \ref{sec:shortexp}) is adopted,
then the new LSST minimum exposure time of 0.1 seconds would easily
accommodate the entire M67 main sequence. Alternatively, or if the
short exposure mode is not adopted, we note that work with the Kepler
mission (e.g., Haas et al. 2011) has shown success using custom pixel
masks to accurately perform photometry on stars as much as 6 magnitudes
brighter than the saturation level. Similar techniques applied to
the LSST fields should enable photometry for the G dwarfs, particularly
those in less-crowded portions of the field.

% ====================================================================
%
% \subsection{Conclusions}
%
% Here we answer the ten questions posed in
% \autoref{sec:intro:evaluation:caseConclusions}:
%
% \begin{description}
%
% \item[Q1:] {\it Does the science case place any constraints on the
% tradeoff between the sky coverage and coadded depth? For example, should
% the sky coverage be maximized (to $\sim$30,000 deg$^2$, as e.g., in
% Pan-STARRS) or the number of detected galaxies (the current baseline
% of 18,000 deg$^2$)?}
%
% \item[A1:] ...
%
% \item[Q2:] {\it Does the science case place any constraints on the
% tradeoff between uniformity of sampling and frequency of  sampling? For
% example, a rolling cadence can provide enhanced sample rates over a part
% of the survey or the entire survey for a designated time at the cost of
% reduced sample rate the rest of the time (while maintaining the nominal
% total visit counts).}
%
% \item[A2:] ...
%
% \item[Q3:] {\it Does the science case place any constraints on the
% tradeoff between the single-visit depth and the number of visits
% (especially in the $u$-band where longer exposures would minimize the
% impact of the readout noise)?}
%
% \item[A3:] ...
%
% \item[Q4:] {\it Does the science case place any constraints on the
% Galactic plane coverage (spatial coverage, temporal sampling, visits per
% band)?}
%
% \item[A4:] ...
%
% \item[Q5:] {\it Does the science case place any constraints on the
% fraction of observing time allocated to each band?}
%
% \item[A5:] ...
%
% \item[Q6:] {\it Does the science case place any constraints on the
% cadence for deep drilling fields?}
%
% \item[A6:] ...
%
% \item[Q7:] {\it Assuming two visits per night, would the science case
% benefit if they are obtained in the same band or not?}
%
% \item[A7:] ...
%
% \item[Q8:] {\it Will the case science benefit from a special cadence
% prescription during commissioning or early in the survey, such as:
% acquiring a full 10-year count of visits for a small area (either in all
% the bands or in a  selected set); a greatly enhanced cadence for a small
% area?}
%
% \item[A8:] ...
%
% \item[Q9:] {\it Does the science case place any constraints on the
% sampling of observing conditions (e.g., seeing, dark sky, airmass),
% possibly as a function of band, etc.?}
%
% \item[A9:] ...
%
% \item[Q10:] {\it Does the case have science drivers that would require
% real-time exposure time optimization to obtain nearly constant
% single-visit limiting depth?}
%
% \item[A10:] ...
%
% \end{description}

\navigationbar


% --------------------------------------------------------------------
